\documentclass[12pt]{article}

\usepackage{geometry}
\usepackage{amsmath}
\usepackage{graphicx}
\usepackage{multicol}

\usepackage{tikz}
\usetikzlibrary{calc,trees,positioning,arrows,fit,shapes,calc}
\usepackage{pgfplots}

\usepackage{longtable}

\newcommand{\ds}{\displaystyle}

\newcommand{\points}[1]{(#1 points.)}		% Trying to be lazy.

\usepackage{array}
\newcolumntype{L}[1]{>{\raggedright\let\newline\\\arraybackslash\hspace{0pt}}m{#1}}
\newcolumntype{C}[1]{>{\centering\let\newline\\\arraybackslash\hspace{0pt}}m{#1}}
\newcolumntype{R}[1]{>{\raggedleft\let\newline\\\arraybackslash\hspace{0pt}}m{#1}}
\newcommand{\red}[1]{\textcolor{red}{#1}}

\topmargin -1in
\textheight 9.5in
\oddsidemargin -0.3in
\evensidemargin \oddsidemargin
\pagestyle{empty}
%\marginparwidth 0.5in
\textwidth 7in
\parindent 0in

%--------------------------------------------------------------------------------------------------------------------------------------------------------------------------
%						Document
%--------------------------------------------------------------------------------------------------------------------------------------------------------------------------


\begin{document}
\pagestyle{plain}

\noindent \textbf{Name:} \underline{\hspace{15em}}		\hfill	Quiz 1 \\
           Math F252X-901, Calculus II  			\hfill	Fall 2024 	

%-------------------------------------------------------------------------------------------------------------
%						Assignment
%-------------------------------------------------------------------------------------------------------------
                \vspace{1cm}
                
24 points possible; each part is worth 2 points. No aids (book, notes,
calculator, phone, etc.) are permitted. Show all work and use proper
notation for full credit. Answers should be in reasonably simplified
form.

\begin{enumerate}
\item \points{10} Compute the derivatives of the following functions.
  \begin{enumerate}
  \item $\ds f(\theta) = \theta \sin(\theta) + \frac{\pi}{4}$
    \vfill
  \item $\ds g(x) = (\cos(3x)+e^x)^4$
    \vfill

  \item $\ds h(x) = \tan(x)\sec(x)$
    \vfill

    \newpage

  \item $\ds y = \frac{\sin(2x)}{x^4+e}$
    \vfill

  \item $\ds G(z) = \ln(z^a - b)$ where $a$ and $b$ are constants
    \vfill

  % \item $\ds F(x) = \arcsin(x^2)$
    % \vfill
  \end{enumerate}
  
  \newpage

\item \points{10} Compute the following antiderivatives (indefinite integrals) and definite integrals.
  \begin{enumerate}
  \item $\ds \int_1^2 \frac{8x^2-4x-6}{2x} \, dx$
    \vfill

  \item $\ds \int 3x^2(x^3 + 4)^7 \, dx$

    \vfill

  \item $\ds \int \frac{\sec^2(x)}{\tan^2(x)} \, dx$
    \vfill

    \newpage

  \item $\ds \int \frac{e^{x}}{1+e^{2x}}$
    \vfill

  \item $\ds \int x(x+1)^5 \, dx$
    \vfill

  \end{enumerate}
  
    
    
\end{enumerate}


\end{document}

%-------------------------------------------------------------------------------------------------------------------------------------------------------------------------------------------------------------------

%%% Local Variables:
%%% mode: latex
%%% TeX-master: t
%%% End:
