\documentclass[12pt]{article}

\usepackage{geometry}
\usepackage{amsmath}
\usepackage{graphicx}
\usepackage{multicol}

\usepackage{tikz}
\usetikzlibrary{calc,trees,positioning,arrows,fit,shapes,calc}
\usepackage{pgfplots}

\usepackage{longtable}

\newcommand{\ds}{\displaystyle}

\newcommand{\points}[1]{(#1 points.)}		% Trying to be lazy.

\usepackage{array}
\newcolumntype{L}[1]{>{\raggedright\let\newline\\\arraybackslash\hspace{0pt}}m{#1}}
\newcolumntype{C}[1]{>{\centering\let\newline\\\arraybackslash\hspace{0pt}}m{#1}}
\newcolumntype{R}[1]{>{\raggedleft\let\newline\\\arraybackslash\hspace{0pt}}m{#1}}
\newcommand{\red}[1]{\textcolor{red}{#1}}

\topmargin -1in
\textheight 9.5in
\oddsidemargin -0.3in
\evensidemargin \oddsidemargin
\pagestyle{empty}
%\marginparwidth 0.5in
\textwidth 7in
\parindent 0in

%--------------------------------------------------------------------------------------------------------------------------------------------------------------------------
%						Document
%--------------------------------------------------------------------------------------------------------------------------------------------------------------------------


\begin{document}
\pagestyle{plain}

\noindent \textbf{Name:} \underline{\hspace{15em}}		\hfill	Quiz 7 \\
           Math F252X-901, Calculus II  			\hfill	Fall 2024 	

%-------------------------------------------------------------------------------------------------------------
%						Assignment
%-------------------------------------------------------------------------------------------------------------
                \vspace{1cm}
                
Thirty minutes maximum. No aids (book, notes,
calculator, phone, etc.) are permitted. Show all work and use proper
notation for full credit. Answers should be in reasonably simplified
form.

\begin{enumerate}

\item \points{5} Write out the first 5 terms of the sequence of
  partial sums for the series $\ds \sum_{n=0}^{\infty} (2n+1)$. \\
  $S_1 =$ \\
  \vspace{0.02cm}
  
  $S_2 =$ \\
  \vspace{0.02cm}

  $S_3 =$ \\
  \vspace{0.02cm}

  $S_4 =$ \\
  \vspace{0.02cm}

  $S_5 =$
  \vspace{0.02cm}

\item Determine whether the following series converge or diverge. If
  the series converges, state its sum. Justify your answers.
  \begin{enumerate}
  \item \points{6} $\ds \sum_{n=1}^{\infty} \frac{1}{3000} \left( \frac{7}{5} \right)^n$.
    \vfill
    
  \item \points{6} $\ds \sum_{n=1}^{\infty} 10 \left( -\frac{3}{5} \right)^n$.
    \vfill
  \end{enumerate}
  
  \newpage

\item What does the divergence test say about the following series?
  Justify your answers.
  \begin{enumerate}
  \item \points{6} $\ds \sum_{n=1}^{\infty} \left( \frac{n}{40n^2+30} \right)$.
    \vfill
    
  \item \points{6} $\ds \sum_{n=1}^{\infty} 9^{(n^{-2})}$.
    \vfill
  \end{enumerate}
  
  
\end{enumerate}


\end{document}

%-------------------------------------------------------------------------------------------------------------------------------------------------------------------------------------------------------------------

%%% Local Variables:
%%% mode: latex
%%% TeX-master: t
%%% End:
