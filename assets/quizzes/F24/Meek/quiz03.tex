\documentclass[12pt]{article}

\usepackage{geometry}
\usepackage{amsmath}
\usepackage{graphicx}
\usepackage{multicol}

\usepackage{tikz}
\usetikzlibrary{calc,trees,positioning,arrows,fit,shapes,calc}
\usepackage{pgfplots}

\usepackage{longtable}

\newcommand{\ds}{\displaystyle}

\newcommand{\points}[1]{(#1 points.)}		% Trying to be lazy.

\usepackage{array}
\newcolumntype{L}[1]{>{\raggedright\let\newline\\\arraybackslash\hspace{0pt}}m{#1}}
\newcolumntype{C}[1]{>{\centering\let\newline\\\arraybackslash\hspace{0pt}}m{#1}}
\newcolumntype{R}[1]{>{\raggedleft\let\newline\\\arraybackslash\hspace{0pt}}m{#1}}
\newcommand{\red}[1]{\textcolor{red}{#1}}

\topmargin -1in
\textheight 9.5in
\oddsidemargin -0.3in
\evensidemargin \oddsidemargin
\pagestyle{empty}
%\marginparwidth 0.5in
\textwidth 7in
\parindent 0in

%--------------------------------------------------------------------------------------------------------------------------------------------------------------------------
%						Document
%--------------------------------------------------------------------------------------------------------------------------------------------------------------------------


\begin{document}
\pagestyle{plain}

\noindent \textbf{Name:} \underline{\hspace{15em}}		\hfill	Quiz 3 \\
           Math F252X-901, Calculus II  			\hfill	Fall 2024 	

%-------------------------------------------------------------------------------------------------------------
%						Assignment
%-------------------------------------------------------------------------------------------------------------
                \vspace{1cm}
                
Thirty minutes maximum. No aids (book, notes,
calculator, phone, etc.) are permitted. Show all work and use proper
notation for full credit. Answers should be in reasonably simplified
form.

\begin{enumerate}

\item Set up integrals to calculate the following values. Do not
  calculate the integrals!
  \begin{enumerate}
    
  \item \points{5} The length of the curve $y=2x^3-\sin(\frac{\pi x}{3})$ on the
    interval $[1,6]$
    \vfill

  \item \points{5} The area of the surface formed by revolving the graph of $y =
    \ln(x)$ on the interval $[2,4]$ around the $x$-axis.
    \vfill

  \item \points{5} The area between the curves $x^2+x$ and $6-x^2$. (Yes, this is
    a review problem.)
    \vfill
    \end{enumerate}

    \newpage

\item Consider the region bounded by the curves $\ds y=e^{-x^2}$, $y=0$,
  $x=1$, and $x=2$.
  \begin{enumerate}
  \item \points{3} Sketch the region.
    \vfill

  \item \points{8}Find the volume of the region obtained by rotating the region
    about the $y$-axis.
    \vfill
  \end{enumerate}

  \newpage

\item \points{4} (BONUS!) Set up an integral to find the area of the surface
  obtained by rotating the region bounded by the curve $y=6x-2x^2$ and
  the $x$-axis about the $y$-axis.
    
\end{enumerate}


\end{document}

%-------------------------------------------------------------------------------------------------------------------------------------------------------------------------------------------------------------------

%%% Local Variables:
%%% mode: latex
%%% TeX-master: t
%%% End:
