\documentclass[12pt]{article}

% Layout.
\usepackage[top=1.2in, bottom=0.75in, left=1in, right=1in, headheight=1.0in, headsep=0pt]{geometry}

% Fonts.
\usepackage{mathptmx}
\usepackage[scaled=0.86]{helvet}
\renewcommand{\emph}[1]{\textsf{\textbf{#1}}}

% TiKZ.
\usepackage{tikz, pgfplots}
\usetikzlibrary{calc}
\pgfplotsset{my style/.append style={axis x line=middle, axis y line=middle, xlabel={$x$}, ylabel={$y$}}}
\pgfplotsset{compat=1.16}

% Misc packages.
\usepackage{amsmath,amssymb,latexsym}
\usepackage{graphicx}
\usepackage{array}
\usepackage{xcolor}
\usepackage{multicol}

% Commands to set various header/footer components.
\makeatletter
\def\doctitle#1{\gdef\@doctitle{#1}}
\doctitle{Use {\tt\textbackslash doctitle\{MY LABEL\}}.}
\def\docdate#1{\gdef\@docdate{#1}}
\docdate{Use {\tt\textbackslash docdate\{MY DATE\}}.}
\def\doccourse#1{\gdef\@doccourse{#1}}
\let\@doccourse\@empty
\def\docscoring#1{\gdef\@docscoring{#1}}
\let\@docscoring\@empty
\def\docversion#1{\gdef\@docversion{#1}}
\let\@docversion\@empty
\makeatother

% Headers and footers layout.
\makeatletter
\usepackage{fancyhdr}
\pagestyle{fancy}
\fancyhf{} % Clears all headers/footers.
\lhead{\emph{\@doctitle\hfill\@docdate} \medskip
\ifnum \value{page} > 1\relax\else\\
\emph{Name: \rule{3.5in}{1pt}\hfill \@docscoring}
\fi}

\rfoot{\emph{\@docversion}}
\lfoot{\emph{\@doccourse}}
\cfoot{\emph{\thepage}}
\renewcommand{\headrulewidth}{0pt}%
\makeatother

% Paragraph spacing
\parindent 0pt
\parskip 6pt plus 1pt

% A problem is a section-like command. Use \problem{5} for a problem worth 5 points.
\newcounter{probcount}
\newcounter{subprobcount}
\setcounter{probcount}{0}

\newcommand{\problem}[1]{%
\par
\addvspace{4pt}%
\setcounter{subprobcount}{0}%
\stepcounter{probcount}%
\makebox[0pt][r]{\emph{\arabic{probcount}.}\hskip1ex}\emph{[#1 points]}\hskip1ex}

\newcommand{\thesubproblem}{\emph{\alph{subprobcount}.}}

% like \problem but with name
\newcommand{\nameprob}[2]{%
\par
\addvspace{4pt}%
\setcounter{subprobcount}{0}%
\stepcounter{probcount}%
\makebox[0pt][r]{\emph{#1.}\hskip1ex}\emph{[#2 points]}\hskip1ex}

% Subproblems are an enumerate-like environment with a consistent
% numbering scheme.  Use \begin{subproblems}\item...\item...\end{subproblems}
\newenvironment{subproblems}{%
\begin{enumerate}%
\setcounter{enumi}{\value{subprobcount}}%
\renewcommand{\theenumi}{\emph{\alph{enumi}}}}%
{\setcounter{subprobcount}{\value{enumi}}\end{enumerate}}

% Blanks for answers in normal and math mode.
\newcommand{\blank}[1]{\rule{#1}{0.75pt}}
\newcommand{\mblank}[1]{\underline{\hspace{#1}}}
\def\emptybox(#1,#2){\framebox{\parbox[c][#2]{#1}{\rule{0pt}{0pt}}}}

% Misc.
\renewcommand{\d}{\displaystyle}
\newcommand{\ds}{\displaystyle}


\doctitle{Math 252 (Bueler): Quiz 10}
\docdate{18 April 2024}
\doccourse{}
\docversion{}
\docscoring{\fbox{{\LARGE \strut}\blank{0.8in} / 25}}

\begin{document}
30 minutes.  No aids (book, notes, calculator, internet, etc.) are permitted.  Show all work and use proper notation for full credit.  Put answers in reasonably-simplified form.  25 points possible.

\newcommand{\threeopts}{{\small \hspace{-6mm} $\begin{matrix} \text{\textsc{converges}} \\ \text{\textsc{absolutely}} \end{matrix}$ \qquad\qquad $\begin{matrix} \text{\textsc{converges}} \\ \text{\textsc{conditionally}} \end{matrix}$ \qquad\qquad \textsc{diverges}} \bigskip}

% like 6.3 #120 (on homework)
\problem{6}  Let {\large $f(x)=\sqrt[3]{x}$}.

\begin{subproblems}
\item Find the first and second Taylor polynomials, of degrees 1 and 2, of $f(x)$ at basepoint $a=8$.
\vfill

\item Use the first Taylor polynomial to estimate {\large $\sqrt[3]{9}$}.
\vfill
\end{subproblems}

% based on 6.3 #131 (on homework)
\problem{3}  We know that {\large $e^x \approx 1 + x + \frac{x^2}{2} + \frac{x^3}{6}$}; this is the 3rd Taylor polynomial at $a=0$.  Evaluate at $-x^2$, and use this to approximate
	$$\int_0^1 e^{-x^2}\,dx \approx \hspace{4.0in}$$
\vfill


\clearpage\newpage
\problem{8}  Let {\large $f(x)=\ln(1+x)$}.

\begin{subproblems}
\item Find the Maclaurin series.  (\textsl{Any valid method is accepted, including from memory.  But get the right series!})
\vspace{3.0in}

\item Use the ratio or root test to find the interval of convergence of the same series.  (\textsl{Hint. Remember to check the endpoints of the interval.})
\vfill
\end{subproblems}


\clearpage\newpage
% like 6.3 #132 on homework, but much simpler
\problem{4}  Let $f(x)=\sin x$ and $a=0$, and consider the interval $[-1,1]$.  Find the smallest value of $n$ so that the remainder estimate $\ds |R_n(x)| \le \frac{M}{(n+1)!} (x-a)^{n+1}$, where $M$ is an upper bound on $|f^{(n+1)}(z)|$ on the interval, yields $\ds |R_n(x)| \le \frac{1}{20}$ on the interval.
\vfill

% compare 6.3 #152; compare 6.3 #155
\problem{4}  Find the Taylor series for {\large $f(x)=x^2$} around $a=1$.
\vfill


\clearpage\newpage
\textbf{\textsf{Extra Credit. [2 points]}} {\large\strut} \, \quad Suppose $f$ is this fifth degree polynomial: $f(x)=1+x+2x^2+3x^3+4x^4+5x^5$.  Write down a \emph{fully simplified} expression for $p_{17}(x)$, the 17th Taylor polynomial of $f(x)$ at basepoint $a=\sqrt{\pi}$.  Explain why your answer, which should require only one line to write, can be written down so immediately.
\vfill

\noindent \hrule
\medskip
\centerline{\footnotesize \textsc{blank space}}
\vfill
\end{document}
