\documentclass[12pt]{article}

% Layout.
\usepackage[top=1.2in, bottom=0.75in, left=1in, right=1in, headheight=1.0in, headsep=0pt]{geometry}

% Fonts.
\usepackage{mathptmx}
\usepackage[scaled=0.86]{helvet}
\renewcommand{\emph}[1]{\textsf{\textbf{#1}}}

% TiKZ.
\usepackage{tikz, pgfplots}
\usetikzlibrary{calc}
\pgfplotsset{my style/.append style={axis x line=middle, axis y line=middle, xlabel={$x$}, ylabel={$y$}}}
\pgfplotsset{compat=1.16}

% Misc packages.
\usepackage{amsmath,amssymb,latexsym}
\usepackage{graphicx}
\usepackage{array}
\usepackage{xcolor}
\usepackage{multicol}

% Commands to set various header/footer components.
\makeatletter
\def\doctitle#1{\gdef\@doctitle{#1}}
\doctitle{Use {\tt\textbackslash doctitle\{MY LABEL\}}.}
\def\docdate#1{\gdef\@docdate{#1}}
\docdate{Use {\tt\textbackslash docdate\{MY DATE\}}.}
\def\doccourse#1{\gdef\@doccourse{#1}}
\let\@doccourse\@empty
\def\docscoring#1{\gdef\@docscoring{#1}}
\let\@docscoring\@empty
\def\docversion#1{\gdef\@docversion{#1}}
\let\@docversion\@empty
\makeatother

% Headers and footers layout.
\makeatletter
\usepackage{fancyhdr}
\pagestyle{fancy}
\fancyhf{} % Clears all headers/footers.
\lhead{\emph{\@doctitle\hfill\@docdate} \medskip
\ifnum \value{page} > 1\relax\else\\
\emph{Name: \rule{3.5in}{1pt}\hfill \@docscoring}
\fi}

\rfoot{\emph{\@docversion}}
\lfoot{\emph{\@doccourse}}
\cfoot{\emph{\thepage}}
\renewcommand{\headrulewidth}{0pt}%
\makeatother

% Paragraph spacing
\parindent 0pt
\parskip 6pt plus 1pt

% A problem is a section-like command. Use \problem{5} for a problem worth 5 points.
\newcounter{probcount}
\newcounter{subprobcount}
\setcounter{probcount}{0}

\newcommand{\problem}[1]{%
\par
\addvspace{4pt}%
\setcounter{subprobcount}{0}%
\stepcounter{probcount}%
\makebox[0pt][r]{\emph{\arabic{probcount}.}\hskip1ex}\emph{[#1 points]}\hskip1ex}

\newcommand{\thesubproblem}{\emph{\alph{subprobcount}.}}

% like \problem but with name
\newcommand{\nameprob}[2]{%
\par
\addvspace{4pt}%
\setcounter{subprobcount}{0}%
\stepcounter{probcount}%
\makebox[0pt][r]{\emph{#1.}\hskip1ex}\emph{[#2 points]}\hskip1ex}

% Subproblems are an enumerate-like environment with a consistent
% numbering scheme.  Use \begin{subproblems}\item...\item...\end{subproblems}
\newenvironment{subproblems}{%
\begin{enumerate}%
\setcounter{enumi}{\value{subprobcount}}%
\renewcommand{\theenumi}{\emph{\alph{enumi}}}}%
{\setcounter{subprobcount}{\value{enumi}}\end{enumerate}}

% Blanks for answers in normal and math mode.
\newcommand{\blank}[1]{\rule{#1}{0.75pt}}
\newcommand{\mblank}[1]{\underline{\hspace{#1}}}
\def\emptybox(#1,#2){\framebox{\parbox[c][#2]{#1}{\rule{0pt}{0pt}}}}

% Misc.
\renewcommand{\d}{\displaystyle}
\newcommand{\ds}{\displaystyle}


\doctitle{Math 252 (Bueler): Quiz 4}
\docdate{8 February 2024}
\doccourse{}
\docversion{}
\docscoring{\fbox{{\LARGE \strut}\blank{0.8in} / 25}}

\begin{document}
30 minutes.  No aids (book, notes, calculator, internet, etc.) are permitted.  Show all work and use proper notation for full credit.  Put answers in reasonably-simplified form.  25 points possible.

\problem{7}  A 2 meter fishing rod is made of solid fiberglass and tapers at the end.  Assume it has a linear mass density function of $\ds \rho(x) = 4 - \frac{x^2}{10000}$ grams per centimeter, where $x=0$ is the thick end.  What is its mass?  Give your answer as a simplified number, with units.
% int 0^200 4 - x^2/10000 = [4x - x^3/30000]_0^200 = 800 - 8000000/30000 = 800 - 800/3 = 1600/3 approx 533 grams = 0.533 kg
\vfill

\problem{10}  Find the derivative, indefinite integral, or definite integral.  Write ``$+C$'' if appropriate.

\begin{subproblems}
\item Find $\ds \frac{dy}{dx}$ if\, $y = \ln(\tan x)$.
\vspace{2.0in}

\clearpage
\newpage

\item $\ds \int_0^{\pi/4} \tan x\,dx = $
\vfill

\item Find $\ds \frac{dy}{dx}$ if\, $y = \log_{10} x$.
\vfill

\item $\ds \int \frac{dx}{x \,\ln x} = $
\vfill

\item Find $\ds \frac{dy}{dx}$ if\, {\large $y = e^{\cos x}$}.  (\textsl{Hint.  Differentiate $\ln y$.})
\vfill
\end{subproblems}

\clearpage\newpage
\problem{8}  It requires 10 Newtons of force to stretch a spring 0.25 m from its natural length.  How much work is required to stretch the spring one meter from its natural length?  Give your answer with units, and in simplified form. (\textsl{Hint. First, what is the spring constant?})
\vfill

\clearpage\newpage
\nameprob{EC}{1} (\emph{Extra Credit})  Assume $a>0$ and $b>0$ are positive numbers.  Simplify both integrals as far as possible.  (\textsl{Credit is given only if both answers are correct and fully simplified.})
\begin{align*}
\int_1^b \frac{1}{t}\,dt &= \hspace{6.0in} \\
\phantom{foo} & \\
\phantom{foo} & \\
\phantom{foo} & \\
\int_a^{ab} \frac{1}{t}\,dt &= 
\end{align*}
\vspace{1.5in}

\noindent \hrule

\bigskip
\centerline{\footnotesize \textsc{blank space}}
\vfill
\end{document}