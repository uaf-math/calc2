\documentclass[12pt]{article}

% Layout.
\usepackage[top=1.2in, bottom=0.75in, left=1in, right=1in, headheight=1.0in, headsep=0pt]{geometry}

% Fonts.
\usepackage{mathptmx}
\usepackage[scaled=0.86]{helvet}
\renewcommand{\emph}[1]{\textsf{\textbf{#1}}}

% TiKZ.
\usepackage{tikz, pgfplots}
\usetikzlibrary{calc}
\pgfplotsset{my style/.append style={axis x line=middle, axis y line=middle, xlabel={$x$}, ylabel={$y$}}}
\pgfplotsset{compat=1.16}

% Misc packages.
\usepackage{amsmath,amssymb,latexsym}
\usepackage{graphicx}
\usepackage{array}
\usepackage{xcolor}
\usepackage{multicol}

% Commands to set various header/footer components.
\makeatletter
\def\doctitle#1{\gdef\@doctitle{#1}}
\doctitle{Use {\tt\textbackslash doctitle\{MY LABEL\}}.}
\def\docdate#1{\gdef\@docdate{#1}}
\docdate{Use {\tt\textbackslash docdate\{MY DATE\}}.}
\def\doccourse#1{\gdef\@doccourse{#1}}
\let\@doccourse\@empty
\def\docscoring#1{\gdef\@docscoring{#1}}
\let\@docscoring\@empty
\def\docversion#1{\gdef\@docversion{#1}}
\let\@docversion\@empty
\makeatother

% Headers and footers layout.
\makeatletter
\usepackage{fancyhdr}
\pagestyle{fancy}
\fancyhf{} % Clears all headers/footers.
\lhead{\emph{\@doctitle\hfill\@docdate} \medskip
\ifnum \value{page} > 1\relax\else\\
\emph{Name: \rule{3.5in}{1pt}\hfill \@docscoring}
\fi}

\rfoot{\emph{\@docversion}}
\lfoot{\emph{\@doccourse}}
\cfoot{\emph{\thepage}}
\renewcommand{\headrulewidth}{0pt}%
\makeatother

% Paragraph spacing
\parindent 0pt
\parskip 6pt plus 1pt

% A problem is a section-like command. Use \problem{5} for a problem worth 5 points.
\newcounter{probcount}
\newcounter{subprobcount}
\setcounter{probcount}{0}

\newcommand{\problem}[1]{%
\par
\addvspace{4pt}%
\setcounter{subprobcount}{0}%
\stepcounter{probcount}%
\makebox[0pt][r]{\emph{\arabic{probcount}.}\hskip1ex}\emph{[#1 points]}\hskip1ex}

\newcommand{\thesubproblem}{\emph{\alph{subprobcount}.}}

% like \problem but with name
\newcommand{\nameprob}[2]{%
\par
\addvspace{4pt}%
\setcounter{subprobcount}{0}%
\stepcounter{probcount}%
\makebox[0pt][r]{\emph{#1.}\hskip1ex}\emph{[#2 points]}\hskip1ex}

% Subproblems are an enumerate-like environment with a consistent
% numbering scheme.  Use \begin{subproblems}\item...\item...\end{subproblems}
\newenvironment{subproblems}{%
\begin{enumerate}%
\setcounter{enumi}{\value{subprobcount}}%
\renewcommand{\theenumi}{\emph{\alph{enumi}}}}%
{\setcounter{subprobcount}{\value{enumi}}\end{enumerate}}

% Blanks for answers in normal and math mode.
\newcommand{\blank}[1]{\rule{#1}{0.75pt}}
\newcommand{\mblank}[1]{\underline{\hspace{#1}}}
\def\emptybox(#1,#2){\framebox{\parbox[c][#2]{#1}{\rule{0pt}{0pt}}}}

% Misc.
\renewcommand{\d}{\displaystyle}
\newcommand{\ds}{\displaystyle}


\doctitle{Math 252 (Bueler): Quiz 7}
\docdate{21 March 2024}
\doccourse{}
\docversion{}
\docscoring{\fbox{{\LARGE \strut}\blank{0.8in} / 25}}

\begin{document}
30 minutes.  No aids (book, notes, calculator, internet, etc.) are permitted.  Show all work and use proper notation for full credit.  Put answers in reasonably-simplified form.  25 points possible.

% 5.1 #10 on homework
\problem{3}  Find a formula for the $n$th term $a_n$ of the sequence whose first several terms are
    $$0,3,8,15,24,35,48,63,80,99,\dots \hspace{4.0in}$$
\vspace{2.0in}

% done in class
\problem{5}  Consider the series \quad $\ds \sum_{n=1}^\infty \frac{1}{n(n+1)}$.

\begin{subproblems}
\item Use partial fractions and ``telescoping'' to write a simplified formula for the partial sum $S_k$.
\vfill

\item Compute the sum of the infinite series.
\vspace{0.7in}
\end{subproblems}


\clearpage\newpage
% 5.1 #27 and #49 from homework
\problem{6}  For each sequence, find the limit if it converges, or show that the sequence diverges.  Indicate any places where you use L'H\^opital's rule.
\begin{subproblems}
\item \quad $\ds \frac{n^2}{2^n}$
\vfill
\item \quad $\ds a_n = \left(1 - \frac{2}{n}\right)^n$
\vfill
\end{subproblems}

\problem{3}  Find a formula for the $n$th term $a_n$ of this recursively-defined sequence, for $n\ge 1$:
    $$a_1 = 3 \quad \text{ and } \quad a_{n+1} = \frac{a_n}{n}. \hspace{4.0in}$$
\vfill


\clearpage\newpage
% 5.2 #90; #91 was on homework
\problem{6}  Does the series converge or diverge?  If it converges find its sum; if it diverges explain why.
\begin{subproblems}
\item \quad $\ds 1 + \frac{e}{\pi} + \frac{e^2}{\pi^2} + \frac{e^3}{\pi^3} + \dots$
\vfill
\item \quad $\ds \sum_{n=1}^\infty \frac{n+1}{n}$
\vfill
\end{subproblems}

\problem{2}  Compute and simplify the partial sum $S_4$ for the series in \emph{5 b.}
\vspace{2.0in}

\clearpage\newpage
\textbf{\textsf{Extra Credit. [1 point]}} {\large\strut} \, \quad The thing below is called the \emph{Sierpinksi gasket}.  It is built by removing the white parts from a black square.  Assume the original square has side-length one and thus area one.  Remove the middle 1/9th of the area.  The remainder is 8 smaller black squares.  For each of these, remove the middle 1/9th.  Continuing in this way forever, you remove all the white area.  Using geometric series, compute the white area you removed.  What area is left, the black area?

\bigskip
\includegraphics[width=0.25\textwidth]{figs/sierp.jpg}
\vspace{2.5in}

\noindent \hrule
\medskip
\centerline{\footnotesize \textsc{blank space}}
\vfill
\end{document}
