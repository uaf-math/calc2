\documentclass[12pt]{article}

% Layout.
\usepackage[top=1.2in, bottom=0.75in, left=1in, right=1in, headheight=1.0in, headsep=0pt]{geometry}

% Fonts.
\usepackage{mathptmx}
\usepackage[scaled=0.86]{helvet}
\renewcommand{\emph}[1]{\textsf{\textbf{#1}}}

% TiKZ.
\usepackage{tikz, pgfplots}
\usetikzlibrary{calc}
\pgfplotsset{my style/.append style={axis x line=middle, axis y line=middle, xlabel={$x$}, ylabel={$y$}}}
\pgfplotsset{compat=1.16}

% Misc packages.
\usepackage{amsmath,amssymb,latexsym}
\usepackage{graphicx}
\usepackage{array}
\usepackage{xcolor}
\usepackage{multicol}

% Commands to set various header/footer components.
\makeatletter
\def\doctitle#1{\gdef\@doctitle{#1}}
\doctitle{Use {\tt\textbackslash doctitle\{MY LABEL\}}.}
\def\docdate#1{\gdef\@docdate{#1}}
\docdate{Use {\tt\textbackslash docdate\{MY DATE\}}.}
\def\doccourse#1{\gdef\@doccourse{#1}}
\let\@doccourse\@empty
\def\docscoring#1{\gdef\@docscoring{#1}}
\let\@docscoring\@empty
\def\docversion#1{\gdef\@docversion{#1}}
\let\@docversion\@empty
\makeatother

% Headers and footers layout.
\makeatletter
\usepackage{fancyhdr}
\pagestyle{fancy}
\fancyhf{} % Clears all headers/footers.
\lhead{\emph{\@doctitle\hfill\@docdate} \medskip
\ifnum \value{page} > 1\relax\else\\
\emph{Name: \rule{3.5in}{1pt}\hfill \@docscoring}
\fi}

\rfoot{\emph{\@docversion}}
\lfoot{\emph{\@doccourse}}
\cfoot{\emph{\thepage}}
\renewcommand{\headrulewidth}{0pt}%
\makeatother

% Paragraph spacing
\parindent 0pt
\parskip 6pt plus 1pt

% A problem is a section-like command. Use \problem{5} for a problem worth 5 points.
\newcounter{probcount}
\newcounter{subprobcount}
\setcounter{probcount}{0}

\newcommand{\problem}[1]{%
\par
\addvspace{4pt}%
\setcounter{subprobcount}{0}%
\stepcounter{probcount}%
\makebox[0pt][r]{\emph{\arabic{probcount}.}\hskip1ex}\emph{[#1 points]}\hskip1ex}

\newcommand{\thesubproblem}{\emph{\alph{subprobcount}.}}

% like \problem but with name
\newcommand{\nameprob}[2]{%
\par
\addvspace{4pt}%
\setcounter{subprobcount}{0}%
\stepcounter{probcount}%
\makebox[0pt][r]{\emph{#1.}\hskip1ex}\emph{[#2 points]}\hskip1ex}

% Subproblems are an enumerate-like environment with a consistent
% numbering scheme.  Use \begin{subproblems}\item...\item...\end{subproblems}
\newenvironment{subproblems}{%
\begin{enumerate}%
\setcounter{enumi}{\value{subprobcount}}%
\renewcommand{\theenumi}{\emph{\alph{enumi}}}}%
{\setcounter{subprobcount}{\value{enumi}}\end{enumerate}}

% Blanks for answers in normal and math mode.
\newcommand{\blank}[1]{\rule{#1}{0.75pt}}
\newcommand{\mblank}[1]{\underline{\hspace{#1}}}
\def\emptybox(#1,#2){\framebox{\parbox[c][#2]{#1}{\rule{0pt}{0pt}}}}

% Misc.
\renewcommand{\d}{\displaystyle}
\newcommand{\ds}{\displaystyle}


\doctitle{Math 252 (Bueler): Quiz 11}
\docdate{25 April 2024}
\doccourse{}
\docversion{}
\docscoring{\fbox{{\LARGE \strut}\blank{0.8in} / 25}}

\begin{document}
30 minutes.  No aids (book, notes, calculator, internet, etc.) are permitted.  Show all work and use proper notation for full credit.  Put answers in reasonably-simplified form.  25 points possible.

\newcommand{\threeopts}{{\small \hspace{-6mm} $\begin{matrix} \text{\textsc{converges}} \\ \text{\textsc{absolutely}} \end{matrix}$ \qquad\qquad $\begin{matrix} \text{\textsc{converges}} \\ \text{\textsc{conditionally}} \end{matrix}$ \qquad\qquad \textsc{diverges}} \bigskip}

% like
\problem{5}  Use binomial series to write the Maclaurin series of {\large $f(x)=\sqrt[3]{1+x}$}.  In particular, write the third Taylor polynomial {\large $p_3(x)$} with simplified coefficients.
\vfill

\problem{4}  Eliminate $t$ from the parametric curve\, {\large $x(t)=5 \cos t$} \, and \, {\large $y(t) = 2\sin t$},\, to write it as a cartesian (rectangular) equation.
\vspace{3.0in}

\clearpage\newpage
\problem{4}  Sketch the parametric curve by eliminating the parameter.  (\textsl{Hint.  Here $t$ can be any real number.  However, pay attention to} which $(x,y)$ \textsl{points are generated by the parametric formula.})
\large
	$$x = e^t, \quad y = e^{2t} \hspace{5.0in}$$
\normalsize
\vfill

\problem{4}  Convert the parametric curve into rectangular form by eliminating the parameter.  No sketch is required.
\large
	$$x = 4t+3, \quad y = 16t^2-9 \hspace{5.0in}$$
\normalsize
\vfill


\clearpage\newpage
\problem{4}  Find the \emph{slope} and the \emph{equation} of the tangent line at \,$t=-1$:
\large
	$$x = 2t, \quad y = t^3 \hspace{5.0in}$$
\normalsize
\vfill

\problem{4}  For the curve \, {\large $x= 4 \cos \theta$} \, and \, {\large $y= 4 \sin \theta$},\, find the concavity at $\theta=\pi/4$.
\vfill


\clearpage\newpage
\textbf{\textsf{Extra Credit. [1 point]}} {\large\strut} \, \quad The parametric curve \, {\large $x= (\arctan t) \cos t$, $y= (\arctan t) \sin t$} \, has a \textsl{circle} as its asymptote as $t\to \infty$.  Find the cartesian equation of this circle.
\vfill

\noindent \hrule
\medskip
\centerline{\footnotesize \textsc{blank space}}
\vfill
\end{document}
