\documentclass[12pt]{article}

% Layout.
\usepackage[top=1.2in, bottom=0.75in, left=1in, right=1in, headheight=1.0in, headsep=0pt]{geometry}

% Fonts.
\usepackage{mathptmx}
\usepackage[scaled=0.86]{helvet}
\renewcommand{\emph}[1]{\textsf{\textbf{#1}}}

% TiKZ.
\usepackage{tikz, pgfplots}
\usetikzlibrary{calc}
%\pgfplotsset{my style/.append style={axis x line=middle, axis y line=middle, xlabel={$x$}, ylabel={$y$}}}
%\pgfplotsset{compat=1.16}

% Misc packages.
\usepackage{amsmath,amssymb,latexsym}
\usepackage{graphicx}
\usepackage{array}
\usepackage{xcolor}
\usepackage{multicol}

% Commands to set various header/footer components.
\makeatletter
\def\doctitle#1{\gdef\@doctitle{#1}}
\doctitle{Use {\tt\textbackslash doctitle\{MY LABEL\}}.}
\def\docdate#1{\gdef\@docdate{#1}}
\docdate{Use {\tt\textbackslash docdate\{MY DATE\}}.}
\def\doccourse#1{\gdef\@doccourse{#1}}
\let\@doccourse\@empty
\def\docscoring#1{\gdef\@docscoring{#1}}
\let\@docscoring\@empty
\def\docversion#1{\gdef\@docversion{#1}}
\let\@docversion\@empty
\makeatother

% Headers and footers layout.
\makeatletter
\usepackage{fancyhdr}
\pagestyle{fancy}
\fancyhf{} % Clears all headers/footers.
\lhead{\emph{\@doctitle\hfill\@docdate}
\ifnum \value{page} > 1\relax\else\\
\emph{Name: \rule{3.5in}{1pt}\hfill \@docscoring}
\fi}

\rfoot{\emph{\@docversion}}
\lfoot{\emph{\@doccourse}}
\cfoot{\emph{\thepage}}
\renewcommand{\headrulewidth}{0pt}%
\makeatother

% Paragraph spacing
\parindent 0pt
\parskip 6pt plus 1pt

% A problem is a section-like command. Use \problem{5} for a problem worth 5 points.
\newcounter{probcount}
\newcounter{subprobcount}
\setcounter{probcount}{0}
\newcommand{\problem}[1]{%
\par
\addvspace{4pt}%
\setcounter{subprobcount}{0}%
\stepcounter{probcount}%
\makebox[0pt][r]{\emph{\arabic{probcount}.}\hskip1ex}\emph{[#1 points]}\hskip1ex}
\newcommand{\thesubproblem}{\emph{\alph{subprobcount}.}}

% Subproblems are an enumerate-like environment with a consistent
% numbering scheme. 
% Use \begin{subproblems}\item...\item...\end{subproblems}
\newenvironment{subproblems}{%
\begin{enumerate}%
\setcounter{enumi}{\value{subprobcount}}%
\renewcommand{\theenumi}{\emph{\alph{enumi}}}}%
{\setcounter{subprobcount}{\value{enumi}}\end{enumerate}}

% Blanks for answers in normal and math mode.
\newcommand{\blank}[1]{\rule{#1}{0.75pt}}
\newcommand{\mblank}[1]{\underline{\hspace{#1}}}
\def\emptybox(#1,#2){\framebox{\parbox[c][#2]{#1}{\rule{0pt}{0pt}}}}

% Misc.
\renewcommand{\d}{\displaystyle}
\newcommand{\ds}{\displaystyle}


\doctitle{Math 252: Quiz 5}
\docdate{28 Sept 2023}
\doccourse{}
\docversion{}
\docscoring{\fbox{{\LARGE \strut}\blank{0.8in} / 25}}

\begin{document}
30 minutes maximum. 25 possible points. No aids (book, calculator, etc.) are permitted  Show all work and use proper notation for full credit.  Answers should be in reasonably-simplified form.\\
Trigonometric Identities\\
\begin{tabular}{ccc}
$\sin^2(x)=\frac{1}{2} ( 1-\cos(2x))$&&$\sin(ax)\cos(bx)=\frac{1}{2} ( \sin((a-b)x)+\sin((a+b)x))$\\

 $\cos^2(x)=\frac{1}{2} ( 1+\cos(2x))$&&$\sin(ax)\sin(bx)=\frac{1}{2} ( \cos((a-b)x)-\cos((a+b)x))$\\
 
 &&$\cos(ax)\cos(bx)=\frac{1}{2} ( \cos((a-b)x)+\cos((a+b)x))$\\
\end{tabular}
\problem{10} Evaluate the definite integrals below:
	\begin{subproblems}
	\item $\displaystyle{\int_{1/3}^{1/2} \cot(\pi x) \: dx}$
	\vfill
	\item $\displaystyle{\int_1^4 \sqrt{x} \ln(x) \: dx}$
	\vfill
	\end{subproblems}



\problem{15} Evaluate the definite integrals
	\begin{subproblems}
	\item $\displaystyle{\int\cos^2(4x) \: dx}$
	\vfill
\newpage

\thispagestyle{empty}

\vspace*{-2cm}

	\item $\displaystyle{\int\arctan(x) \: dx}$
	\vfill
	\item $\displaystyle{\int\tan^3(x)\sec^4(x) \: dx}$
	\vfill
	\item $\displaystyle{\int\frac{dx}{x \ln(x)}}$
	\vfill
	\item $\displaystyle{\int x^2 \cos(x) \: dx}$
	\vfill
	\end{subproblems}

\end{document}