\documentclass[12pt]{article}

\usepackage{geometry}
\usepackage{amsmath}
\usepackage{graphicx}
\usepackage{multicol}

\usepackage{tikz}
\usetikzlibrary{calc,trees,positioning,arrows,fit,shapes,calc}
\usepackage{pgfplots}

\usepackage{longtable}

\newcommand{\ds}{\displaystyle}

\newcommand{\points}[1]{(#1 points.)}		% Trying to be lazy.

\usepackage{array}
\newcolumntype{L}[1]{>{\raggedright\let\newline\\\arraybackslash\hspace{0pt}}m{#1}}
\newcolumntype{C}[1]{>{\centering\let\newline\\\arraybackslash\hspace{0pt}}m{#1}}
\newcolumntype{R}[1]{>{\raggedleft\let\newline\\\arraybackslash\hspace{0pt}}m{#1}}
\newcommand{\red}[1]{\textcolor{red}{#1}}

\topmargin -1in
\textheight 9.5in
\oddsidemargin -0.3in
\evensidemargin \oddsidemargin
\pagestyle{empty}
%\marginparwidth 0.5in
\textwidth 7in
\parindent 0in

%--------------------------------------------------------------------------------------------------------------------------------------------------------------------------
%						Document
%--------------------------------------------------------------------------------------------------------------------------------------------------------------------------


\begin{document}
\pagestyle{plain}

\noindent \textbf{Name:} \underline{\hspace{15em}}		\hfill	Quiz 1 \\
           Math F252X-902, Calculus II  			\hfill	Spring 2025 	

%-------------------------------------------------------------------------------------------------------------
%						Assignment
%-------------------------------------------------------------------------------------------------------------
                \vspace{1cm}
                
16 points possible; each part is worth 2 points. No aids (book, notes,
calculator, phone, etc.) are permitted. Show all work and use proper
notation for full credit. Answers should be in reasonably simplified
form.

\begin{enumerate}
\item Compute the derivatives of the following functions.
  \begin{enumerate}
  \item $\ds f(x) = x \ln(x) + \frac{\pi}{4}$
    \vfill

  \item $\ds h(\theta) = \sin(\theta)\sec(\theta)$ (Simplify as much
    as possible.)
    \vfill

    \newpage

  \item $\ds y = \frac{e^{(2x)}}{x^4+e}$ (Do not simplify.)
    \vfill

  \item $\ds G(z) = \sin(z^a - b)$ where $a$ and $b$ are constants
    \vfill

  % \item $\ds F(x) = \arcsin(x^2)$
    % \vfill
  \end{enumerate}
  
  \newpage

\item Compute the following antiderivatives (indefinite integrals) and definite integrals.
  \begin{enumerate}
  \item $\ds \int_0^2 \frac{x}{x^2+1} \, dx$
    \vfill

  \item $\ds \int \frac{\csc^2(x)}{\cot^2(x)} \, dx$
    \vfill

    \newpage

  \item $\ds \int \frac{1+e^{2x}-e^{5x}}{e^{x}} \, dx$
    \vfill

  \item $\ds \int x \sqrt{x+3} \, dx$
    \vfill

  \end{enumerate}
  
    
    
\end{enumerate}


\end{document}

%-------------------------------------------------------------------------------------------------------------------------------------------------------------------------------------------------------------------

%%% Local Variables:
%%% mode: latex
%%% TeX-master: t
%%% End:
