\documentclass[12pt]{article}

\usepackage{geometry}
\usepackage{amsmath}
\usepackage{graphicx}
\usepackage{multicol}

\usepackage{tikz}
\usetikzlibrary{calc,trees,positioning,arrows,fit,shapes,calc}
\usepackage{pgfplots}

\usepackage{longtable}

\newcommand{\ds}{\displaystyle}

\newcommand{\points}[1]{(#1 points.)}		% Trying to be lazy.

\usepackage{array}
\newcolumntype{L}[1]{>{\raggedright\let\newline\\\arraybackslash\hspace{0pt}}m{#1}}
\newcolumntype{C}[1]{>{\centering\let\newline\\\arraybackslash\hspace{0pt}}m{#1}}
\newcolumntype{R}[1]{>{\raggedleft\let\newline\\\arraybackslash\hspace{0pt}}m{#1}}
\newcommand{\red}[1]{\textcolor{red}{#1}}

\topmargin -1in
\textheight 9.5in
\oddsidemargin -0.3in
\evensidemargin \oddsidemargin
\pagestyle{empty}
%\marginparwidth 0.5in
\textwidth 7in
\parindent 0in

%--------------------------------------------------------------------------------------------------------------------------------------------------------------------------
%						Document
%--------------------------------------------------------------------------------------------------------------------------------------------------------------------------


\begin{document}
\pagestyle{plain}

\noindent \textbf{Name:} \underline{\hspace{15em}}		\hfill	Quiz 7 \\
           Math F252X-902, Calculus II  			\hfill	Spring 2025 	

%-------------------------------------------------------------------------------------------------------------
%						Assignment
%-------------------------------------------------------------------------------------------------------------
                \vspace{1cm}
                
Graded out of 40 points. No aids (book, notes,
calculator, phone, etc.) are permitted. Show all work and use proper
notation for full credit. Answers should be in reasonably simplified
form.

\begin{enumerate}

\item Consider the sequence $a_1 = 0$, $a_n = 2a_{n-1} + 1$.
  \begin{enumerate}
  \item \points{4} Write out the first four terms of this sequence.
    \vfill

  \item \points{6} Find an explicit formula for the $n$th term of this sequence.
    \vfill
  \end{enumerate}

\item Consider the sequence $\ds \left\{ \frac{1}{2}, \frac{2}{3},
    \frac{3}{4}, \frac{4}{5}, \frac{5}{6}, \dots \right\}$.

  \begin{enumerate}
  \item \points{4} Find an explicit formula for the $n$th term of this sequence.
    \vfill

  \item \points{6} Determine whether the sequence converges or diverges. If the
    sequence converges, find its limit. {\bf Justify your answer!}
    \vfill
  \end{enumerate}

  \newpage

\item Consider the series $\ds \sum_{n=1}^{\infty} \left( \sqrt{n} -
    \sqrt{n+1} \right) $.
  \begin{enumerate}
  \item \points{4} Write out the first four partial sums:
    \vspace{0.3cm}

    $S_1 = $
    \vspace{0.3cm}

    $S_2 = $
    \vspace{0.3cm}

    $S_3 = $
    \vspace{0.3cm}

    $S_4 = $
    \vspace{0.3cm}

  \item \points{6} Determine whether the series converges or diverges. If it
    converges, find its limit. {\bf Justify your answer!}
    \vfill

  \end{enumerate}

\item Consider the series $1 + e + e^2 + e^3 + \dots$.
  \begin{enumerate}
  \item \points{4} Rewrite the series using summation notation.
    \vfill

  \item \points{6} Determine whether the series converges or diverges. If it
    converges, find its limit. {\bf Justify your answer!}
    \vfill
  \end{enumerate}
  
  \newpage

\item \points{2} BONUS: Determine whether the series $\ds
  \sum_{n=1}^{\infty} \frac{1}{n(n+1)}$ converges or diverges. If it
  converges, find its limit. {\bf Justify your answer!}
  
\end{enumerate}


\end{document}

%-------------------------------------------------------------------------------------------------------------------------------------------------------------------------------------------------------------------

%%% Local Variables:
%%% mode: latex
%%% TeX-master: t
%%% End:
