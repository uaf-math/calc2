\documentclass[12pt]{article}

\usepackage{geometry}
\usepackage{amsmath}
\usepackage{graphicx}
\usepackage{multicol}

\usepackage{tikz}
\usetikzlibrary{calc,trees,positioning,arrows,fit,shapes,calc}
\usepackage{pgfplots}

\usepackage{longtable}

\newcommand{\ds}{\displaystyle}

\newcommand{\points}[1]{(#1 points.)}		% Trying to be lazy.

\usepackage{array}
\newcolumntype{L}[1]{>{\raggedright\let\newline\\\arraybackslash\hspace{0pt}}m{#1}}
\newcolumntype{C}[1]{>{\centering\let\newline\\\arraybackslash\hspace{0pt}}m{#1}}
\newcolumntype{R}[1]{>{\raggedleft\let\newline\\\arraybackslash\hspace{0pt}}m{#1}}
\newcommand{\red}[1]{\textcolor{red}{#1}}

\topmargin -1in
\textheight 9.5in
\oddsidemargin -0.3in
\evensidemargin \oddsidemargin
\pagestyle{empty}
%\marginparwidth 0.5in
\textwidth 7in
\parindent 0in

%--------------------------------------------------------------------------------------------------------------------------------------------------------------------------
%						Document
%--------------------------------------------------------------------------------------------------------------------------------------------------------------------------


\begin{document}
\pagestyle{plain}

\noindent \textbf{Name:} \underline{\hspace{15em}}		\hfill	Quiz 2 \\
           Math F252X-902, Calculus II  			\hfill	Spring 2025 	

%-------------------------------------------------------------------------------------------------------------
%						Assignment
%-------------------------------------------------------------------------------------------------------------
                \vspace{1cm}
                
Graded out of 22 points. No aids (book, notes,
calculator, phone, etc.) are permitted. Show all work and use proper
notation for full credit. Answers should be in reasonably simplified
form.

\begin{enumerate}
    
\item Consider the curves $y=5-x$ and $y=7-x^2$.
  \begin{enumerate}
  \item \points{4} Sketch the region bounded by the curves.
    \vfill

  \item \points{6} Find the area of the region bounded by the curves.
    \vfill
    \vfill
  \end{enumerate}
  
  \newpage

\item
  \begin{enumerate}
  \item \points{4} Sketch the region bounded by the curves $y=x^2$,
    $y=-x+2$, and the $x$-axis.
    \vfill

  \item \points{4} Set up an integral to find the volume of the solid
    of revolution obtained by rotating the region from part (a) about the
    $x$-axis. Do not evaluate the integral.
    \vfill

  \item \points{4} Set up an integral to find the volume of the
    solid of revolution obtained by rotating the region from part (a)
    about the $y$-axis. Do not evaluate the integral.
    \vfill
  \end{enumerate}

  \newpage

\item BONUS \points{3} Consider the region bounded by $y=x^2+3$ and
  $y=x+5$. Find the volume of the solid of revolution obtained by
  rotating this region about the line $y=3$.
 
\end{enumerate}


\end{document}

%-------------------------------------------------------------------------------------------------------------------------------------------------------------------------------------------------------------------

%%% Local Variables:
%%% mode: latex
%%% TeX-master: t
%%% End:
