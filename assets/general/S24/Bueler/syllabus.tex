\documentclass[12pt]{article}

% Layout.
\usepackage[top=1.2in, bottom=0.9in, left=1in, right=1in, headheight=1in, headsep=6pt]{geometry}

% Fonts.
\usepackage{mathptmx}
\usepackage[scaled=1.0]{helvet}
\renewcommand{\emph}[1]{\textsf{\textbf{#1}}}

% Misc packages.
\usepackage{amsmath,amssymb,latexsym}
\usepackage{graphicx,hyperref}
\usepackage{array}
\usepackage{xcolor}
\usepackage{multicol}
\usepackage{tabularx,colortbl}
\usepackage{enumitem}

\hypersetup{
    colorlinks=true,
    linkcolor=blue,
    filecolor=magenta,      
    urlcolor=blue,
    pdftitle={Syllabus for MATH F252X section 001 Fall 2022 (Bueler)},
    pdfpagemode=FullScreen,
    }

\def\mailto#1{\href{mailto:#1}{#1}}

% Paragraph spacing
\parindent 0pt
\parskip 6pt plus 1pt
\def\tableindent{\hskip 0.5 in}
\def\ts{\hskip 1.5 em}

\usepackage{fancyhdr}
\pagestyle{fancy} 
\lhead{\large\sf\textbf{MATH F252 Calculus II}}
\chead{\large\sf\textbf{Syllabus}}
\rhead{\large\sf\textbf{Fall 2022}}

\newcommand{\localhead}[1]{\par\smallskip\textbf{#1} \smallskip\nobreak\\}%
\def\heading#1{\localhead{\large\emph{#1}}}
\def\subheading#1{\localhead{\emph{#1}}}

\newenvironment{clist}%
{\bgroup\parskip 0pt\begin{list}{$\bullet$}{\partopsep 4pt\topsep 0pt\itemsep -2pt}}%
{\end{list}\egroup}%

\begin{document}

\strut\par\vskip-12pt
\heading{Essential Information}

\vskip -12pt
\strut\hbox to \hsize{\tableindent\vtop{\halign{#\hfill\ts&#\hfil\cr
{\emph{Instructor}} & Ed Bueler \quad \href{mailto:elbueler@alaska.edu}{\texttt{elbueler\@@alaska.edu}} \cr
\strut & \cr
{\emph{Course webpage}}&\href{https://bueler.github.io/calc2/}{\texttt{bueler.github.io/calc2/}}\cr
\strut & \cr
{\emph{Canvas site}} & \href{https://canvas.alaska.edu/courses/9933}{\texttt{canvas.alaska.edu/courses/9933}} \cr
\strut & \cr
\emph{Prerequisite} & MATH F251X Calculus I; or placement.\cr
\strut & \cr
{\emph{Required text}} &\textit{OpenStax Calculus Volume 2} by G. Strang \& E. Herman,\cr
&\href{https://openstax.org/details/books/calculus-volume-2}{\texttt{openstax.org/details/books/calculus-volume-2}} \cr
&
  optional paperback print copy:\quad \texttt{ISBN-13 978-1-50669-807-6}\cr
  }
\hfil}}


\heading{Description and Course Goals}
Calculus is useful, and part of the language, in all the science and engineering disciplines.  That is why it is part of the UAF core curriculum.  The two principal tools from Calculus I (MATH 251), differentiation and integration, are central here too, but this course extends your understanding of integration especially.  We will also study sequences and series, including Taylor series.

At the completion of the course, students will:
\begin{clist}
\item possess mature skills in computing integrals in one variable,
\item be able to apply integrals to common problems (areas, volumes, work, \dots),
\item understand the convergence of sequences and series,
\item understand and be able to apply Taylor series and polynomials, and
\item be able to use parametric and polar equations for curves.
\end{clist}
Upon completion, students will have the mathematical foundation for success in Calculus III and other courses requiring mathematics background, including many junior/senior-level science and engineering courses.  


\heading{Class Time}
There are 4.5 hours of class meetings with your instructor every week:
\begin{itemize}
\item MWF 9:15 am -- 10:15 pm  Gruening 409
\item Th 9:45 am -- 11:15 pm  Gruening 409
\end{itemize}

The 1.5 hour time on Thursday will be used to discuss new topics (30 minutes) and for a weekly Quiz (30 minutes).  The final 30 minutes, at most, will be spent going over the Quiz.  We will go over the quiz immediately so that students can leave knowing what Quiz material has been mastered, how to work each problem correctly, and what topics need additional work. 

\clearpage\newpage

\strut \vspace{-12pt}

\heading{Tuesday 9:45--10:45} 
For Tuesday, we will not use the official Recitation (MATH F252L) classroom.  Instead, UAF Math Services (\href{https://uaf.edu/dms/mathlab/}{\texttt{uaf.edu/dms/mathlab/}}) has scheduled group tutoring specifically for this class during that time.  A zoom link for this online tutoring will be posted at the \href{https://canvas.alaska.edu/courses/9933}{Canvas page}.


\heading{Schedule and Online Materials}
The course website contains a \href{https://bueler.github.io/calc2/schedule.pdf}{Schedule} listing the textbook sections to be covered each class, the dates each Homework is due, plus the dates for Quizzes and Exams. You should consult this schedule frequently.  I will announce any Schedule adjustments in class.

Most course materials (Syllabus, old Quiz and Exam solutions, study materials, etc.) will be posted on the \href{https://bueler.github.io/calc2/}{course webpage}.  Some course materials (grades, Homework solutions, announcements, etc.) will be available on the \href{https://canvas.alaska.edu/courses/9933}{Canvas site}.  Each website links to the other.


\heading{Office Hours and Communication}
My Office Hours are online at \href{http://bueler.github.io/OffHrs.htm}{\texttt{bueler.github.io/OffHrs.htm}}.  Students can also schedule meetings with me outside of regular office hours, or mail me at \href{mailto:elbueler@alaska.edu}{\texttt{elbueler\@@alaska.edu}}.

I will use Canvas to send announcements.  If I need to contact you outside of class times, I'll try to email via Canvas.  Please set the email address in Canvas to one that you check regularly!




\heading{Evaluation and Grades}
Grades are determined as follows.  (Each component of the grade is discussed below.)
 
\begin{multicols}{2}
\begin{tabular}{|c|c|}
\hline
3 Minute Questions & 5\%\\
\hline
Homework & 10\% \\
\hline
Quizzes & 20\% \\
\hline
Midterm Exam 1 & 20\% \\
\hline
Midterm Exam 2 & 20\%  \\
\hline
Final Exam & 25\% \\
\hline
total & 100\% \, \\
\hline
\end{tabular}

%\vskip 6pt

\begin{tabular}{llll}
A  & 93--100\%& C  & 68--75\%  \\
A- & 90--92\% & C- & not given \\
B+ & 87--89\% & D+ & 65--67\%  \\
B  & 82--86\% & D  & 60--65\%  \\
B- & 79--81\% & D- & 57--59\%  \\
C+ & 76--78\% & F  & $\le$ 56\%
\end{tabular}
\end{multicols}

The grade ranges at right are a guarantee and a lower bound. I reserve the right to increase your grade above these ranges based on the actual difficulty of the work and/or on average class performance. Any such increases will preserve grade ordering by weighted total score. 


\heading{3 Minute Questions and Participation}
Attendance and participation are important parts of mastering the material, and a strong predictor of overall course performance in any subject.

Instead of using an attendance sheet, at the start of two classes/lectures per week you will find a ``3 Minute Question,'' a half sheet of paper, on your desk.  Please answer the question, write your name on it, and turn it in.  Naturally, you'll have 3 minutes to do this after the start of class!  It will not be graded for content at all, but whether you did it will be recorded in lieu of attendance.

Please let me know by email (\href{mailto:elbueler@alaska.edu}{\texttt{elbueler\@@alaska.edu}}) if you will miss a class for any reason.


\heading{Homework}
Homework assignments consist of a selection of problems from the textbook.  Please write your Homework on paper, or electronically on a tablet, and turned in as a PDF via Gradescope, accessed via the \href{https://canvas.alaska.edu/courses/9933}{Canvas page}.  Help with scanning homework can be found on the \href{https://uaf-math251.github.io/techHelp.html}{Tech Help} webpage.  Assignments are due, mostly on Mondays and Wednesdays, by 11:59pm, thus in advance of the Thursday Quiz; see the online \href{https://bueler.github.io/calc2/schedule.pdf}{Schedule} for due dates.  The list of Homework problems is at the \href{https://bueler.github.io/calc2/homework.html}{Homework} tab on the public webpage.

Complete worked solutions to all Homework problems are \emph{provided in advance on the Canvas site}!  Therefore Homework will be graded based on \emph{effort} and \emph{completion}.  All students should earn 100\% on homework.  Of course, it is possible, and pathetic, to defeat the purpose of the homework by copying the solutions.  This is a bad idea, and because Homework is only 10\% of your grade it is not worthwhile.  The Homework exists so you can \emph{learn by doing}.


\heading{Quizzes}
A weekly Quiz will be given on Thursdays.  It will be in the middle third of the 1.5 hour, 9:45--11:15 class.  A Quiz will cover material since the previous Quiz.  Quizzes are given under Exam conditions: books, notes, and calculators are not allowed.  Performance on Quizzes is your best indicator of how well you are learning the course material, and it is a much better predictor of Exam performance than is your Homework score.

Students will be given the opportunity to grade and correct their Quizzes in the last third of the Thursday class.  You can earn-back up to half the missed points for doing so \emph{accurately}.

Always contact me if you will miss a Quiz for a justified reason!  There are enough Quizzes during the semester so that make-up Quizzes will normally not be created.  Instead I will do something like asking a coach to proctor your Quiz if you are on the road, or I will mark it as excused. 


\heading{Midterm and Final Exams}
There are two Midterm Exams this semester, to be held on the dates in the schedule on the course website: \emph{Midterm Exam 1 on Thursday October 6} and \emph{Midterm Exam 2 on Thursday November 17}.  Midterms are given during the class time.

Make-up Midterms will be given only for documented extenuating circumstances, at my discretion.

The cumulative Final Exam will be held at the day/time listed as the common math time in the online schedule: \textbf{10:15-12:15 Tuesday December 13}.  The Final will be in our regular classroom, Gruening 409.  Department policy does not allow me to give an early Final Exam.

All Exams will be closed book, closed notes, and no calculator.


\heading{Tutoring and Resources}
\vskip -30pt\strut
\begin{clist}
    \item As mentioned above, on Tuesdays there will be group tutoring specifically for calculus 2.  A zoom link for this online tutoring is posted at the \href{https://canvas.alaska.edu/courses/9933}{Canvas page}.
	\item The Math and Stat Lab, Chapman Building Room 305, offers tutors. 
	See 

	\href{http://www.uaf.edu/dms/mathlab/}{\texttt{www.uaf.edu/dms/mathlab/}} for schedules and availability.
	\item Free
one-on-one (or small group) tutoring is available in 
Chapman Building Room 201. You must schedule an
appointment; see \href{http://www.uaf.edu/dms/mathlab/}{\texttt{www.uaf.edu/dms/mathlab/}}.
	\item Student Support Services (\href{https://uaf.edu/sss/}{\texttt{uaf.edu/sss/}}) offers free tutoring in many subjects to students who qualify for their program.
	\item ASUAF (\href{https://uaf.edu/asuaf/}{\texttt{uaf.edu/asuaf/}}) offers private tutoring for a small fee, based on student income.
\end{clist}

\heading{Rules and Policies}
\vskip -20pt

\subheading{Incomplete Grade} 
Incomplete (I) will only be given in
  DMS courses in cases where
  the student has completed the majority (normally all but the last
  three weeks) of a course with a grade of C or better, but for
  personal reasons beyond his/her control has been unable to complete
  the course during the regular term. Negligence or indifference are
  not acceptable reasons for the granting of an incomplete. 

\subheading{Late Withdrawals} 
A withdrawal after the deadline from a DMS course will
  normally be granted only in cases where the student is performing
  satisfactorily (i.e., C or better) in a course, but has exceptional
  reasons, beyond his/her control, for being unable to complete the
  course. These exceptional reasons should be detailed in writing to
  the instructor, department head and dean.

\subheading{No Early Final Examinations}
Final examinations for DMS
  courses shall not be held earlier than the date and time published
  in the official term schedule. Normally, a student will not be
  allowed to take a final exam early. Exceptions can be made by
  individual instructors, but should only be allowed in exceptional
  circumstances and in a manner which doesn't endanger the security of
  the exam.

\subheading{Academic Dishonesty}
Academic dishonesty, including cheating and plagiarism, will not
be tolerated.  It is a violation of the Student Code of Conduct
and will be punished according to UAF procedures.

%\begin{center} \textsc{Syllabus Addendum} \end{center}

\subheading{Student protections and services statement}
Every qualified student is welcome in my classroom.  As needed, I am happy to work with you, Disability Services, Veterans' Services, Rural Student Services, etc.~to find reasonable accommodations.  Students at this University are protected against sexual harassment and discrimination (Title IX), and minors have additional protections.  As required, if I notice or am informed of certain types of misconduct, then I am required to report it to the appropriate authorities.  For more information on your rights as a student and the resources available to you to resolve problems, please go the following site: \href{https://www.uaf.edu/handbook/}{\texttt{www.uaf.edu/handbook/}}.


\subheading{General education statement}
This course is listed as a General Education Math Course.  As such this course is expected to \textsl{contribute to the student meeting}\footnote{This phrase was added to the GER statement because it is nonsensical without it.} the 4 general learning outcomes. 

\begin{enumerate}
\item Build knowledge of human institutions, sociocultural processes, and the physical and natural works through the study of mathematics.  Competence will be demonstrated for the foundational information in each subject area, its context and significance, and the methods used in advancing each.

\item Develop intellectual and practical skills across the curriculum, including inquiry and analysis, critical and creative thinking, problem solving, written and oral communication, information literacy, technological competence, and collaborative learning. Proficiency will be demonstrated across the curriculum through critical analysis of proffered information, well-reasoned solutions to problems or inferences drawn from evidence, effective written and oral communication, and satisfactory outcomes of group projects.

\item Acquire tools for effective civic engagement in local through global contexts, including ethical reasoning, intercultural competence, and knowledge of Alaska and Alaska issues.  Facility will be demonstrated through analyses of issues including dimensions of ethics, human and cultural diversity, conflicts and interdependencies, globalization, and sustainability.   

\item Integrate and apply learning, including synthesis and advanced accomplishment across general and specialized studies, adapting them to new settings, questions and responsibilities, and forming a foundation for lifelong learning. Preparation will be demonstrated though production of a a creative or scholarly product that requires broad knowledge, appropriate technical proficiency, information collection, synthesis, interpretation, presentation and reflection.
\end{enumerate}

\hfill  \scriptsize [syllabus version: \today] \normalsize

\end{document}
