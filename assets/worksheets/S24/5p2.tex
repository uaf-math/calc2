\documentclass[11pt,fleqn]{article} 
\usepackage[margin=0.8in, head=0.8in]{geometry} 
\usepackage{amsmath, amssymb, amsthm}
\usepackage{fancyhdr} 
\usepackage{palatino, url, multicol}
\usepackage{graphicx, pgfplots} 
\usepackage[all]{xy}
\usepackage{polynom} 
%\usepackage{pdfsync} %% I don't know why this messes up tabular column widths
\usepackage{enumerate}
\usepackage{framed}
\usepackage{setspace}
\usepackage{array,tikz}

\pgfplotsset{compat=1.6}

\pgfplotsset{soldot/.style={color=black,only marks,mark=*}} \pgfplotsset{holdot/.style={color=black,fill=white,only marks,mark=*}}

\newcommand{\ds}{\displaystyle}
\pagestyle{fancy} 
\lfoot{}
\rfoot{\S 5.2}

\begin{document}
\renewcommand{\headrulewidth}{0pt}
\newcommand{\blank}[1]{\rule{#1}{0.75pt}}
\newcommand{\bc}{\begin{center}}
\newcommand{\ec}{\end{center}}


\vspace*{-0.7in}

%%%%%%%%%intro page
\begin{center}
  \large
  \sc{Section 5.2: \,Series}\\
   
\end{center}

\footnotesize
Things to know by the end of this section
\begin{multicols}{2}
	\begin{enumerate}[a.]
	\item how to use sigma notation \emph{with facility}
	\item the meaning of a \emph{series}, especially as compared to a \emph{sequence} (from \S 5.1)
	\item the meaning of \emph{a sequence of partial sums of a series} and how to find it.
	\item what it means to say a series converges.
	\item what a \emph{geometric series} is and how to determine whether or not it converges.
	\item what a \emph{telescoping series} is and how to determine whether or not it converges.	
	\end{enumerate}
\end{multicols}

\normalsize
\vspace{-5mm}
\hrulefill
\begin{enumerate}
\item An infinite series is
    $$\sum_{n=1}^\infty a_n = a_1 + a_2 + a_3 + \dots$$
The sequence of its partial sums is

\vspace{1.5in}

\item For each series below, expand the sigma notation and then \emph{compute and simplify the first 4 partial sums} $S_1, S_2, S_3,S_4.$  (Use a calculating device to get a decimal, if desired.) 
	\begin{enumerate}
	\item $\displaystyle \sum_{n=1}^\infty \left( \frac{2}{3} \right)^n$
	\vfill
	\item $\displaystyle \sum_{n=1}^\infty  \frac{1}{n(n+1)} $
	\vfill
	\item $\displaystyle \sum_{n=1}^\infty  \frac{(-1)^n}{5}$
	\vfill

\clearpage\newpage
	\item $\displaystyle \sum_{n=1}^\infty  \frac{n}{n^2+2}$
	\vfill
	\end{enumerate}

\item \emph{Complete the bulleted lines below.} \quad \textbf{Definition:} The series $\ds \sum_{n=1}^\infty a_n$ \\
\begin{itemize}
	\item converges if \\
	
	\item diverges if \\
	
\end{itemize}

\item The series in 2(a) is a geometric series.  Its $k$th partial sum is $\ds S_k = \sum_{n=1}^k \left(\frac{2}{3}\right)^n$.  Below, write out $S_k$ without sigma notation, multiply by $2/3$, then subtract, and then cancel as many terms as possible:
\begin{align*}
S_k &= \hspace{5.0in} \\ \\
\frac{2}{3} S_k &= \\ \\ \hline
\left(1 - \frac{2}{3}\right) S_k = \frac{1}{3} S_k &=
\end{align*}
This has led to a closed formula for $S_k$:
	$$S_k = \hspace{5.0in}$$
Therefore the infinite series converges to \quad $\ds \sum_{n=1}^\infty \left(\frac{2}{3}\right)^n = \lim_{k\to\infty} S_k = \boxed{\phantom{\int fosd asdf sdfa}}$.

\item Do the same for the geometric series in 2(c).

\vspace{1.7in}
\end{enumerate}
\end{document}

