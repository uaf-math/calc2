\documentclass[11pt,fleqn]{article} 
\usepackage[margin=0.8in, head=0.8in]{geometry} 
\usepackage{amsmath, amssymb, amsthm}
\usepackage{fancyhdr} 
\usepackage{palatino, url, multicol}
\usepackage{graphicx, pgfplots} 
\usepackage[all]{xy}
\usepackage{polynom} 
%\usepackage{pdfsync} %% I don't know why this messes up tabular column widths
\usepackage{enumerate}
\usepackage{framed}
\usepackage{setspace}
\usepackage{array,tikz}

\pgfplotsset{compat=1.6}

\pgfplotsset{soldot/.style={color=black,only marks,mark=*}} \pgfplotsset{holdot/.style={color=black,fill=white,only marks,mark=*}}


\pagestyle{fancy} 
\lfoot{}
\rfoot{\S 3.1}

\begin{document}
\renewcommand{\headrulewidth}{0pt}
\newcommand{\blank}[1]{\rule{#1}{0.75pt}}
\newcommand{\bc}{\begin{center}}
\newcommand{\ec}{\end{center}}
\newcommand{\ds}{\displaystyle}

\vspace*{-0.7in}

%%%%%%%%%intro page
\begin{center}
  \large
  \sc{Section 3.4: Integration by Partial Fractions}\\
   
\end{center}

\begin{enumerate}
\item Express the rational function as a sum of simpler rational functions.  That is: \textbf{expand in partial fractions}.

	\begin{enumerate}
	\item \emph{like 3.4 \#182} \qquad $\ds \frac{2}{(x-1)(x-3)} = $
	\vfill
	
	\item \emph{3.4 \#183} \qquad $\ds \frac{x^2 + 1}{x(x+1)(x+2)} = $
	\vfill
	
	\item \emph{3.4 \#188} \qquad $\ds \frac{1}{(x-1)(x^2+1)} = $
	\vfill
	\end{enumerate}

\item \textbf{Evaluate the integrals using partial fractions}.

	\begin{enumerate}
	\item \emph{3.4 \#204} \qquad $\ds \int \frac{2}{x^2 - x - 6}\,dx = $
	\vfill

\clearpage\newpage
	\item \emph{like 3.4 \#211} \qquad $\ds \int \frac{x+3}{(x^2+1)(x-4)}\,dx = $
	\vfill

	\item \emph{like 3.4 \#203} \qquad $\ds \int_1^2 \frac{2-x}{x^2+x}\,dx = $
	\vfill

	\item \emph{3.4 \#227}; \emph{hint: start with a substitution} \qquad $\ds \int \frac{1}{1+e^x}\,dx =$
	\vfill
	\end{enumerate}
\end{enumerate}
\end{document}
