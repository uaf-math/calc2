
\documentclass[11pt,fleqn]{article} 
\usepackage[margin=0.8in, head=0.8in]{geometry} 
\usepackage{amsmath, amssymb, amsthm}
\usepackage{fancyhdr} 
\usepackage{palatino, url, multicol}
\usepackage{graphicx, pgfplots} 
\usepackage[all]{xy}
\usepackage{polynom} 
%\usepackage{pdfsync} %% I don't know why this messes up tabular column widths
\usepackage{enumerate}
\usepackage{framed}
\usepackage{setspace}
\usepackage{array,tikz}

\pgfplotsset{compat=1.6}

\pgfplotsset{soldot/.style={color=black,only marks,mark=*}} \pgfplotsset{holdot/.style={color=black,fill=white,only marks,mark=*}}


\pagestyle{fancy} 
\lfoot{}
\rfoot{MI Review}

\begin{document}
\renewcommand{\headrulewidth}{0pt}
\newcommand{\blank}[1]{\rule{#1}{0.75pt}}
\newcommand{\bc}{\begin{center}}
\newcommand{\ec}{\end{center}}
\renewcommand{\d}{\displaystyle}

\vspace*{-0.7in}

%%%%%%%%%intro page
\begin{center}
  \large
  \sc{Review for Midterm I}\\
   
\end{center}
{\Large{Basics}}\\

\begin{itemize}
\item Thursday 9:45-11:15
\item Bring a $ 3 \times 5 $ notecard. Writing front and back.
\item Cell phones should be either in a zippered pocket of backpack or facedown on your desk, not in your pocket or lap. Smart watches should be in backpack.
\item You will be required to sit where there is an exam (ie spread out).
\item There will be two versions of the exam.\\
\end{itemize}

\noindent{\Large{Sections Covered}}\\

\begin{itemize}
\item Section 2.1 Areas between Curves\\
Given a region defined by various familiar curves set up and evaluate a definite integral that calculates the area of the region.
\item Section 2.2 Volumes by Slicing (disks, washers or defined cross-sections)
	\begin{itemize}
	\item Given the base and a characterization of the cross-sections of an object, find its volume.
	\item Given a region $R$ in the $xy$-plane and an axis or rotation ($x$- or $y$-axis), determine the volume using disks or washers.
	\end{itemize}
\item Section 2.3 Volumes of Revolution using Shells\\
Given a region $R$ in the $xy$-plane and an axis or rotation ($x$- or $y$-axis), determine the volume using shells.
\item Section 2.4 Arc Length of a Curve and Surface Area
	\begin{itemize}
	\item Given a curve $C$ in the $xy$-plane, find the arc length of $C$. 
	\item Given a curve $C$ in the $xy$-plane and an axis of rotation ($x$- or $y$-axis), find the surface area of the volume of rotation.
	\end{itemize}
\item Section 2.5 Physical Applications
	\begin{itemize}
	\item Work. Including Hooke's Law and pumping fluid out of a container.
	\item Mass.
	\end{itemize}
\item Section 2.6 Moments and Centers of Mass\\
Find the center of mass of 1- and 2-dimensional regions.
\item Section 2.7 A Second Look at Exponential and Logarithmic Functions\\
Nothing from this section will appear directly. You should know how to integrate and differentiate exponential and logarithmic functions as you did in Calc I.

\newpage

\item Chapter 3 Integration Techniques.\\
For each technique, you should be able to recognize what integrals are suitable to the technique and know how to implement this technique.
	\begin{itemize}
	\item Section 3.1 Integration by Parts
	\item Section 3.2 Trigonometric Integrals
	\item Section 3.3 Trigonometric Substitution
	\item Section 3.4 Partial Fractions
	\end{itemize}
\end{itemize}


\noindent{\Large{Other Considerations}}\\

\begin{itemize}
\item Employ good test-taking skills including moving on from any problem that is either taking too long or seems to have (temporarily) stumped you. 
\item Walk in with the assurance that every problem is something doable with the skills taught in the class and learned in homework, quizzes, and/or in-class worksheets.
\item There will be integrals that you are expected to set up but not evaluate. You need to read the directions for the problems and follow them. 
\item You are expected to reasonably simplify your answers, including both numerical answers and integrals you are only required to set up.
\item Be prepared to use all of the integration techniques we have learned.
\item You work, including its organization, matters.
\item Don't write things you know are incorrect but don't leave anything completely blank. 

\end{itemize}


\end{document}

