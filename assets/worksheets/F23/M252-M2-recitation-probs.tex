
\documentclass[11pt,fleqn]{article} 
\usepackage[margin=0.8in, head=0.8in]{geometry} 
\usepackage{amsmath, amssymb, amsthm}
\usepackage{fancyhdr} 
\usepackage{palatino, url, multicol}
\usepackage{graphicx, pgfplots} 
\usepackage[all]{xy}
\usepackage{polynom} 
%\usepackage{pdfsync} %% I don't know why this messes up tabular column widths
\usepackage{enumerate}
\usepackage{framed}
\usepackage{setspace}
\usepackage{array,tikz}

\pgfplotsset{compat=1.6}

\pgfplotsset{soldot/.style={color=black,only marks,mark=*}} \pgfplotsset{holdot/.style={color=black,fill=white,only marks,mark=*}}


\pagestyle{fancy} 
\lfoot{}
\rfoot{M2 practice}

\begin{document}
\renewcommand{\headrulewidth}{0pt}
\newcommand{\blank}[1]{\rule{#1}{0.75pt}}
\newcommand{\bc}{\begin{center}}
\newcommand{\ec}{\end{center}}
\renewcommand{\d}{\displaystyle}

\vspace*{-0.7in}

%%%%%%%%%intro page
\begin{center}
  \large
  \sc{Midterm II Practice Problems}\\
   
\end{center}
\begin{enumerate}
\item Indefinite Integrals
	\begin{enumerate}
	\item The directions for all such problems include the words \emph{Use correct notation.} What does this mean in practice?
	\item Compute and simplify the improper integral $\d \int_2^5 \frac{3x}{x^2-4} \: dx$ or show that it diverges. Use correct notation.
	\item Check your \textbf{work} (not just your answer) against the solution. Did you use correct notation? Could anyone other than you follow and understand your work?
	\end{enumerate}
\item Sequences and Series: The Basics\\
	
	Let $a_n=\frac{n}{2n+1}$ for $n=1,2,3,\cdots$
		\begin{enumerate}
		\item Write the first four terms in the \textbf{sequence} $a_1,a_2,a_3, \cdots.$
		\item Does the \textbf{sequence} above converge? Show your work.
		\item Write the \textbf{series} $\quad \frac{1}{3}+\frac{2}{5}+\frac{3}{7}+\frac{4}{9}+ \cdots$ using sum ($\sum$) notation.
		\item Does the \textbf{series} above converge? Show that your answer is correct by stating the test you are using and applying that test.
		\item Write $S_1, S_2,$ and $S_3,$ the first three terms in the sequence of partial sums of the \textbf{series}. Does the \textbf{sequence} $S_1, S_2, S_3, S_4, \cdots$ converge? Explain your answer.
		\end{enumerate}
	
\item Convervent and Divergent Series
	\begin{itemize}
	\item Expect to use \emph{different} tests. No midterm or final will be do-able by repeatedly using the same one or two tests over and over again.
	\item Recall that a correct answer here is \emph{not} ``converge" or ``diverge" since a coin can get that right 50\% of the time. A correct answer is \textbf{a correct application of an appropriate test.} So the act of checking your answer should not be limited to the words ``converge" or ``diverge.''
	\item Check you \textbf{work.} Does it look anything like the solutions? Could anyone other than you follow and understand your work?
	\end{itemize}
	
	Determine if the following series converge or diverge. Show your work.
	
	\begin{enumerate}
	\item $\d \sum_{n=0}^\infty \frac{(-1)^n}{3\sqrt{n}+\pi}$
	\item $\d \sum_{n=1}^\infty \frac{(-1)^n}{n^e}$
	\item $\d \sum_{n=1}^\infty \frac{5}{n+\ln(n)}$
	\item $\d \sum_{n=1}^\infty \frac{10^n}{(2n)!}$
	\item $\d \sum_{n=0}^\infty \frac{n2^n}{5^n}$
	\item $\d \sum_{n=2}^\infty \frac{\sin^3(n)}{n^2+1}$
	\end{enumerate}
\item Use the Integral Test to show that the series $\d \sum_{n=0}^\infty \frac{1}{1+n^2}$ converges.
\item Explain why the series $\d \sum_{n=2}^\infty \frac{(-2)^n}{7^n}$ converges and determine its sum. (HINT: Look carefully at the summation limits)
\item Find the interval of convergence for the power series 
	$\d \sum_{n=0}^\infty \frac{1}{3^n}(x-1)^n$
\item Find a power series representation for $\frac{2x}{3+x}$ 	
\item Write the Taylor series for $f(x)=e^{2x}$ centered at $x=1.$
\end{enumerate}
\end{document}

