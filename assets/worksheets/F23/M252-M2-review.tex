
\documentclass[11pt,fleqn]{article} 
\usepackage[margin=0.8in, head=0.8in]{geometry} 
\usepackage{amsmath, amssymb, amsthm}
\usepackage{fancyhdr} 
\usepackage{palatino, url, multicol,hyperref}
\usepackage{graphicx, pgfplots} 
\usepackage[all]{xy}
\usepackage{polynom} 
%\usepackage{pdfsync} %% I don't know why this messes up tabular column widths
\usepackage{enumerate}
\usepackage{framed}
\usepackage{setspace}
\usepackage{array,tikz}

\pgfplotsset{compat=1.6}

\pgfplotsset{soldot/.style={color=black,only marks,mark=*}} \pgfplotsset{holdot/.style={color=black,fill=white,only marks,mark=*}}


\pagestyle{fancy} 
\lfoot{}
\rfoot{MII Review}

\begin{document}
\renewcommand{\headrulewidth}{0pt}
\newcommand{\blank}[1]{\rule{#1}{0.75pt}}
\newcommand{\bc}{\begin{center}}
\newcommand{\ec}{\end{center}}
\renewcommand{\d}{\displaystyle}

\vspace*{-0.7in}

%%%%%%%%%intro page
\begin{center}
  \large
  \sc{Review for Midterm II}\\
   
\end{center}
{\Large{Basics}}\\

\begin{itemize}
\item Thursday 9:45-11:15
\item The textbook summary of tests will be attached to the midterm. \href{https://openstax.org/books/calculus-volume-2/pages/5-6-ratio-and-root-tests}{Link} to textbook. \href{https://uaf-math251.github.io/calc2/assets/worksheets/F23/convsummary.pdf}{Link} in course webpage.
\item Cell phones should be either in a zippered pocket of a backpack or facedown on your desk, not in your pocket or lap. Smart watches should be in a backpack.
\item You will be required to sit where there is an exam (ie spread out).
\item There will be two versions of the exam.\\
\end{itemize}

\noindent{\Large{Sections Covered}}\\

\begin{itemize}
\item Section 3.7 Improper Integrals\\

How to recognize and evaluate an improper integral. How to determine if an improper integral converges or diverges.\\

Practical Notes:
	\begin{itemize}
	\item The first step of understanding an improper integral is to rewrite it in terms of a limit.
	\item You must complete the integration and substitution prior to evaluating the limit.
	\item If you \emph{formally} use the method of $u$-substitution, it is safer to resubstitute prior to evaluating the limit.\\
	\end{itemize}

\item Section 5.1 Sequences\\

	\begin{enumerate}
	\item Understand the difference between a \emph{sequence} and a \emph{series}. 
	\item Know how to write the terms of a sequence whether the terms are given via an explicit formula or a recursive one. 
	\item Know how to write a formula for a sequence given term-by-term (i.e. look for a pattern and  generalize it).\\
	\end{enumerate}

\item Section 5.2 Infinite Series\\

	\begin{enumerate}
	\item Know what is meant by \emph{the sequence of partial sums} of a series and be able to find a few of the terms.
	\item Know that a series converges if and only if its sequence of partial sums converges. (This is the definition of convergence for series.)
	\item Know what a geometric series is and how to determine when it converges and when it diverges. If it converges, know to what it converges. 
	\item Know that the \textbf{harmonic series} is and that it diverges.
	\item Know how to identify and exploit the properties of \emph{telescoping series}.
	\end{enumerate}

\item Section 5.3 The Divergence and Integral Tests\\

	\begin{enumerate}
	\item How to use the Divergence Test and know its limits.
	\item How to use the Integral Test.
	\item Know what is meant by a \textbf{$p$-series} and under what conditions a $p$-series converges and diverges.
	\item Any questions about remainders will be extra credit.
	\end{enumerate}

\item Section 5.4 Comparison Tests\\

	\begin{enumerate}
	\item Know \textbf{how} and when to apply the (direct) comparison test and the limit comparison test.\\
	\end{enumerate}

\item Section 5.5 Alternating Series \\

	\begin{enumerate}
	\item Know what is meant by an alternating series.
	\item Know how to apply the Alternating Series Test.
	\item Know how to estimate the remainder of a convergent alternating series when the sum of the series is estimated by a partial sum.
	\item Know what is meant by \textbf{absolute convergence} and \textbf{conditional convergence}.
	\item Know that absolute convergence implies convergence and why this fact is useful.
	\item Know what you must do to \emph{show} that a series is absolutely convergent.
	\item Know what you must do to \emph{show} that a series is conditionally convergent.\\
	\end{enumerate}

\item Section 5.6 Ratio and Root Tests\\

	\begin{enumerate}
	\item Know how to apply the ratio test.
	\item Know how to apply the root test.
	\item Keep in mind that for the root test, you may need to  be somewhat careful about your algebra when taking limits.
	\item Know how to work with factorials.
	\end{enumerate}
	
\textbf{Summary:} You need to think about what characteristics of series suggest one test or another. \\

\textbf{Cautionary Notes:}
	\begin{itemize}
	\item You cannot ever assert a series converges or diverges without a justification.
	\item The only series for which the justification does \emph{not} require the application of a formal test are geometric series, $p$-series, and the harmonic series.
	\end{itemize}

\item Section 6.1 Power Series and Functions\\

	\begin{enumerate}
	\item Know what is meant by a power series centered at $x=a.$
	\item Know how to find the radius of convergence and the center of convergence.
	\item Know the power series representation of $f(x)=\frac{1}{1-x}$ and how to use it to find power series representations of other similar functions.
	\end{enumerate}

\item Section 6.2 Properties of Power Series\\

	\begin{enumerate}
	\item Know how to operate on power series term-by-term within their radius of convergence including addition, multiplication, differentiation and integration.
	\end{enumerate}

\item Section 6.3 Working with Taylor Series\\

	\begin{enumerate}
	\item Know how to find the Taylor Series of a function $f(x)$ centered at $x=a.$
	\item Know how to find the $n$th Taylor polynomial, $p_n$ centered at $x=a.$
	\item You should know that a Maclaurin series is a Taylor series with center $a=0.$
	\item Remainder questions will only appear as extra credit.
	\end{enumerate}

\end{itemize}


\end{document}

