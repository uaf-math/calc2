\documentclass[11pt]{amsart}
%prepared in AMSLaTeX, under LaTeX2e
\addtolength{\oddsidemargin}{-1.2in} 
\addtolength{\evensidemargin}{-1.2in}
\addtolength{\topmargin}{-.9in}
\addtolength{\textwidth}{2.0in}
\addtolength{\textheight}{1.5in}

\renewcommand{\baselinestretch}{1.05}

\usepackage{verbatim} % for "comment" environment

\usepackage{palatino}

\usepackage[final]{graphicx}

\usepackage{tikz}
\usetikzlibrary{positioning}

\usepackage{enumitem,xspace,fancyvrb}

\newtheorem*{thm}{Theorem}
\newtheorem*{defn}{Definition}
\newtheorem*{example}{Example}
\newtheorem*{problem}{Problem}
\newtheorem*{remark}{Remark}

\DefineVerbatimEnvironment{mVerb}{Verbatim}{numbersep=2mm,frame=lines,framerule=0.1mm,framesep=2mm,xleftmargin=4mm,fontsize=\footnotesize}

% macros
\usepackage{amssymb}
\newcommand{\bA}{\mathbf{A}}
\newcommand{\bB}{\mathbf{B}}
\newcommand{\bE}{\mathbf{E}}
\newcommand{\bF}{\mathbf{F}}
\newcommand{\bJ}{\mathbf{J}}

\newcommand{\bb}{\mathbf{b}}
\newcommand{\br}{\mathbf{r}}
\newcommand{\bv}{\mathbf{v}}
\newcommand{\bw}{\mathbf{w}}
\newcommand{\bx}{\mathbf{x}}

\newcommand{\hbi}{\mathbf{\hat i}}
\newcommand{\hbj}{\mathbf{\hat j}}
\newcommand{\hbk}{\mathbf{\hat k}}
\newcommand{\hbn}{\mathbf{\hat n}}
\newcommand{\hbr}{\mathbf{\hat r}}
\newcommand{\hbt}{\mathbf{\hat t}}
\newcommand{\hbx}{\mathbf{\hat x}}
\newcommand{\hby}{\mathbf{\hat y}}
\newcommand{\hbz}{\mathbf{\hat z}}
\newcommand{\hbphi}{\mathbf{\hat \phi}}
\newcommand{\hbtheta}{\mathbf{\hat \theta}}
\newcommand{\complex}{\mathbb{C}}
\newcommand{\ppr}[1]{\frac{\partial #1}{\partial r}}
\newcommand{\ppt}[1]{\frac{\partial #1}{\partial t}}
\newcommand{\ppx}[1]{\frac{\partial #1}{\partial x}}
\newcommand{\ppy}[1]{\frac{\partial #1}{\partial y}}
\newcommand{\ppz}[1]{\frac{\partial #1}{\partial z}}
\newcommand{\pptheta}[1]{\frac{\partial #1}{\partial \theta}}
\newcommand{\ppphi}[1]{\frac{\partial #1}{\partial \phi}}
\newcommand{\Div}{\ensuremath{\nabla\cdot}}
\newcommand{\Curl}{\ensuremath{\nabla\times}}
\newcommand{\curl}[3]{\ensuremath{\begin{vmatrix} \hbi & \hbj & \hbk \\ \partial_x & \partial_y & \partial_z \\ #1 & #2 & #3 \end{vmatrix}}}
\newcommand{\cross}[6]{\ensuremath{\begin{vmatrix} \hbi & \hbj & \hbk \\ #1 & #2 & #3 \\ #4 & #5 & #6 \end{vmatrix}}}
\newcommand{\eps}{\epsilon}
\newcommand{\grad}{\nabla}
\newcommand{\ip}[2]{\ensuremath{\left<#1,#2\right>}}
\newcommand{\lam}{\lambda}
\newcommand{\lap}{\triangle}

\newcommand{\Null}{\operatorname{null}}
\newcommand{\rank}{\operatorname{rank}}
\newcommand{\range}{\operatorname{range}}
\newcommand{\trace}{\operatorname{tr}}

\newcommand{\RR}{\mathbb{R}}
\newcommand{\ZZ}{\mathbb{Z}}

\newcommand{\prob}[1]{\bigskip\noindent\textbf{#1.}\quad }
\newcommand{\exer}[2]{\prob{Exercise #2 on page #1}}
\newcommand{\exerpages}[2]{\prob{Exercise #2 on pages #1}}

\newcommand{\pts}[1]{(\emph{#1 pts}) }
\newcommand{\epart}[1]{\bigskip\noindent\textbf{(#1)}\quad }
\newcommand{\ppart}[1]{\,\textbf{(#1)}\quad }

\newcommand{\Matlab}{\textsc{Matlab}\xspace}

\newcommand{\ds}{\displaystyle}

\begin{document}

%%%%%%%%%intro page
\begin{center}
  \large
  \sc{Section 7.3: Polar Coordinates (day 2)}\\
   
\end{center}

\begin{enumerate}
\item The points below are in polar coordinates. Convert them to rectangular coordinates.
	\begin{enumerate}
	\item $\ds (5, 5 \pi/3)$\\
	\vfill
	\item $\ds (-0.5, -5 \pi /6)$\\
	\vfill
	\end{enumerate}
\item The points below are in rectangular coordinates. Convert them to polar coordinates.
	\begin{enumerate}
	\item $\ds (-8,8)$\\
	\vfill
	\item $\ds (2\sqrt{3},2)$\\
	\vfill
	\end{enumerate}
\item Describe the graph of each polar equation below and convert it to a rectangular equation.	
	\begin{enumerate}
	\item $\ds r=4$\\
	\vfill
	\item $\ds r=5\csc(\theta)$\\
	\vfill
	\end{enumerate}
\newpage
\item Convert the equations below from rectangular equations to polar equations.
 	\begin{enumerate}
	\item $\ds x^2+y^2=20$\\
	\vfill
	\item $\ds y=5x^2$\\
	\vfill
	\end{enumerate}
\item Sketch the graph of the polar equations below.
	\begin{enumerate}
	\item $\ds r=3-2\sin(\theta)$
	
\begin{tikzpicture}[>=latex, scale=.8]
% Draw the lines at multiples of pi/12
\foreach \ang in {0,...,31} {
  \draw [lightgray] (0,0) -- (\ang * 180 / 12:4);
}
% Concentric circles and radius labels
\foreach \s in {0, 1, 2, 3} {
  \draw [lightgray] (0,0) circle (\s + 0.5);
  \draw (0,0) circle (\s);
  \node [fill=white] at (\s, 0) [below] {\scriptsize $\s$};
}
% Add the labels at multiples of pi/4
\foreach \ang/\lab/\dir in {
  0/0/right,
  1/{\pi/4}/{above right},
  2/{\pi/2}/above,
  3/{3\pi/4}/{above left},
  4/{\pi}/left,
  5/{5\pi/4}/{below left},
  7/{7\pi/4}/{below right},
  6/{3\pi/2}/below} {
  \draw (0,0) -- (\ang * 180 / 4:4.1);
  \node [fill=white] at (\ang * 180 / 4:4.2) [\dir] {\scriptsize $\lab$};
}
% Add the labels at multiples of pi/6
\foreach \ang/\lab/\dir in {
  1/{\pi/6}/{above right},
  2/{\pi/3}/above,
  4/{2\pi/3}/{above left},
  5/{5\pi/6}/{left},
  7/{7\pi/6}/{below left},
  8/{4\pi/3}/below,
  10/{5\pi/3}/below,
  11/{11\pi/6}/right} {
  \draw (0,0) -- (\ang * 180 / 6:4.1);
  \node [fill=white] at (\ang * 180 / 6:4.2) [\dir] {\scriptsize $\lab$};
}
% The double-lined circle around the whole diagram
\draw [style=double] (0,0) circle (4);
\end{tikzpicture}

\item $\ds r=2 \cos (3 \theta)$

\begin{tikzpicture}[>=latex, scale=.8]
% Draw the lines at multiples of pi/12
\foreach \ang in {0,...,31} {
  \draw [lightgray] (0,0) -- (\ang * 180 / 12:4);
}
% Concentric circles and radius labels
\foreach \s in {0, 1, 2, 3} {
  \draw [lightgray] (0,0) circle (\s + 0.5);
  \draw (0,0) circle (\s);
  \node [fill=white] at (\s, 0) [below] {\scriptsize $\s$};
}
% Add the labels at multiples of pi/4
\foreach \ang/\lab/\dir in {
  0/0/right,
  1/{\pi/4}/{above right},
  2/{\pi/2}/above,
  3/{3\pi/4}/{above left},
  4/{\pi}/left,
  5/{5\pi/4}/{below left},
  7/{7\pi/4}/{below right},
  6/{3\pi/2}/below} {
  \draw (0,0) -- (\ang * 180 / 4:4.1);
  \node [fill=white] at (\ang * 180 / 4:4.2) [\dir] {\scriptsize $\lab$};
}
% Add the labels at multiples of pi/6
\foreach \ang/\lab/\dir in {
  1/{\pi/6}/{above right},
  2/{\pi/3}/above,
  4/{2\pi/3}/{above left},
  5/{5\pi/6}/{left},
  7/{7\pi/6}/{below left},
  8/{4\pi/3}/below,
  10/{5\pi/3}/below,
  11/{11\pi/6}/right} {
  \draw (0,0) -- (\ang * 180 / 6:4.1);
  \node [fill=white] at (\ang * 180 / 6:4.2) [\dir] {\scriptsize $\lab$};
}
% The double-lined circle around the whole diagram
\draw [style=double] (0,0) circle (4);
\end{tikzpicture}
	\end{enumerate}

	
\end{enumerate}

\end{document}
\begin{tikzpicture}[>=latex, scale=.8]
% Draw the lines at multiples of pi/12
\foreach \ang in {0,...,31} {
  \draw [lightgray] (0,0) -- (\ang * 180 / 12:4);
}
% Concentric circles and radius labels
\foreach \s in {0, 1, 2, 3} {
  \draw [lightgray] (0,0) circle (\s + 0.5);
  \draw (0,0) circle (\s);
  \node [fill=white] at (\s, 0) [below] {\scriptsize $\s$};
}
% Add the labels at multiples of pi/4
\foreach \ang/\lab/\dir in {
  0/0/right,
  1/{\pi/4}/{above right},
  2/{\pi/2}/above,
  3/{3\pi/4}/{above left},
  4/{\pi}/left,
  5/{5\pi/4}/{below left},
  7/{7\pi/4}/{below right},
  6/{3\pi/2}/below} {
  \draw (0,0) -- (\ang * 180 / 4:4.1);
  \node [fill=white] at (\ang * 180 / 4:4.2) [\dir] {\scriptsize $\lab$};
}
% Add the labels at multiples of pi/6
\foreach \ang/\lab/\dir in {
  1/{\pi/6}/{above right},
  2/{\pi/3}/above,
  4/{2\pi/3}/{above left},
  5/{5\pi/6}/{left},
  7/{7\pi/6}/{below left},
  8/{4\pi/3}/below,
  10/{5\pi/3}/below,
  11/{11\pi/6}/right} {
  \draw (0,0) -- (\ang * 180 / 6:4.1);
  \node [fill=white] at (\ang * 180 / 6:4.2) [\dir] {\scriptsize $\lab$};
}
% The double-lined circle around the whole diagram
\draw [style=double] (0,0) circle (4);
\end{tikzpicture}
\hfill
\begin{tikzpicture}[>=latex, scale=.8]
% Draw the lines at multiples of pi/12
\foreach \ang in {0,...,31} {
  \draw [lightgray] (0,0) -- (\ang * 180 / 12:4);
}
% Concentric circles and radius labels
\foreach \s in {0, 1, 2, 3} {
  \draw [lightgray] (0,0) circle (\s + 0.5);
  \draw (0,0) circle (\s);
  \node [fill=white] at (\s, 0) [below] {\scriptsize $\s$};
}
% Add the labels at multiples of pi/4
\foreach \ang/\lab/\dir in {
  0/0/right,
  1/{\pi/4}/{above right},
  2/{\pi/2}/above,
  3/{3\pi/4}/{above left},
  4/{\pi}/left,
  5/{5\pi/4}/{below left},
  7/{7\pi/4}/{below right},
  6/{3\pi/2}/below} {
  \draw (0,0) -- (\ang * 180 / 4:4.1);
  \node [fill=white] at (\ang * 180 / 4:4.2) [\dir] {\scriptsize $\lab$};
}
% Add the labels at multiples of pi/6
\foreach \ang/\lab/\dir in {
  1/{\pi/6}/{above right},
  2/{\pi/3}/above,
  4/{2\pi/3}/{above left},
  5/{5\pi/6}/{left},
  7/{7\pi/6}/{below left},
  8/{4\pi/3}/below,
  10/{5\pi/3}/below,
  11/{11\pi/6}/right} {
  \draw (0,0) -- (\ang * 180 / 6:4.1);
  \node [fill=white] at (\ang * 180 / 6:4.2) [\dir] {\scriptsize $\lab$};
}
% The double-lined circle around the whole diagram
\draw [style=double] (0,0) circle (4);
\end{tikzpicture}


\end{enumerate}
\end{document}
