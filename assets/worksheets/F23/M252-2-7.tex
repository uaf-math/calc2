
\documentclass[11pt,fleqn]{article} 
\usepackage[margin=0.8in, head=0.8in]{geometry} 
\usepackage{amsmath, amssymb, amsthm}
\usepackage{fancyhdr} 
\usepackage{palatino, url, multicol}
\usepackage{graphicx, pgfplots} 
\usepackage[all]{xy}
\usepackage{polynom} 
%\usepackage{pdfsync} %% I don't know why this messes up tabular column widths
\usepackage{enumerate}
\usepackage{framed}
\usepackage{setspace}
\usepackage{array,tikz}

\pgfplotsset{compat=1.6}

\pgfplotsset{soldot/.style={color=black,only marks,mark=*}} \pgfplotsset{holdot/.style={color=black,fill=white,only marks,mark=*}}


\pagestyle{fancy} 
\lfoot{}
\rfoot{\S 2.7}

\begin{document}
\renewcommand{\headrulewidth}{0pt}
\newcommand{\blank}[1]{\rule{#1}{0.75pt}}
\newcommand{\bc}{\begin{center}}
\newcommand{\ec}{\end{center}}
\renewcommand{\d}{\displaystyle}

\vspace*{-0.7in}

%%%%%%%%%intro page
\begin{center}
  \large
  \sc{Section 2.7: Integrals, Exponential Functions and Logarithms}\\
   
\end{center}

\begin{enumerate}
\item List things you know about the function $f(x)=\ln(x)$.
\vspace{1.5in}
\item A new definition for the natural logarithm.
\vspace{1.5in}
\item Explain/justify how the facts below follow immediately from this definition.
	\begin{enumerate}	
	\item $\ln(1)=0.$
	\vfill
	\item If $0 < x <1,$ then $\ln(x) < 0.$
	\vfill
	\item The domain of $f(x)=\ln(x)$ is restricted to positive $x$-values.
	\vfill
	\item The graph of $f(x)=\ln(x)$ keeps growing but is grows at a slower and slower rate.
	\vfill
	\item $\frac{d}{dx} \left( \ln(x) \right) = \frac{1}{x}.$
	\vfill
	\end{enumerate}
\newpage
\item Another way to discover logarithm rules.
\vspace{1.5in}
\item Another view of the number $e$ and the function $g(x)=e^x.$
\vfill
\item Use this definition (and rules about logarithms) to confirm the rule $e^pe^q=e^{p+q}.$
\vfill
\item Use the fact that $N=e^{\ln(N)}$ provided $N>0,$ to find the derivative of $y=a^x$ for $a>0.$
\vfill
\end{enumerate}
\end{document}

