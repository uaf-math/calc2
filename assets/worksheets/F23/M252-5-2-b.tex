\documentclass[11pt,fleqn]{article} 
\usepackage[margin=0.8in, head=0.8in]{geometry} 
\usepackage{amsmath, amssymb, amsthm}
\usepackage{fancyhdr} 
\usepackage{palatino, url, multicol}
\usepackage{graphicx, pgfplots} 
\usepackage[all]{xy}
\usepackage{polynom} 
%\usepackage{pdfsync} %% I don't know why this messes up tabular column widths
\usepackage{enumerate}
\usepackage{framed}
\usepackage{setspace}
\usepackage{array,tikz}

\pgfplotsset{compat=1.6}

\pgfplotsset{soldot/.style={color=black,only marks,mark=*}} \pgfplotsset{holdot/.style={color=black,fill=white,only marks,mark=*}}

\newcommand{\ds}{\displaystyle}
\pagestyle{fancy} 
\lfoot{}
\rfoot{\S 5.2}

\begin{document}
\renewcommand{\headrulewidth}{0pt}
\newcommand{\blank}[1]{\rule{#1}{0.75pt}}
\newcommand{\bc}{\begin{center}}
\newcommand{\ec}{\end{center}}


\vspace*{-0.7in}

%%%%%%%%%intro page
\begin{center}
  \large
  \sc{Section 5.2: Series (Day 2)}\\
   
\end{center}

\textbf{NOTE:} The symbol  \textbf{!!!} indicates that this series is one of the top three series to understand. These series will be used repeatedly in this and other classes.

\begin{enumerate}
\item \textbf{(!!!)} A geometric series has form 

\vspace{1 in}

\item \textbf{Ex 1:} $\displaystyle \sum_{n=1}^\infty \left( \frac{2}{3} \right)^{n-1}$

\vfill

\item \textbf{Ex 2:} $\displaystyle \sum_{n=1}^\infty  \frac{4^{n-1}}{3^{n}} $

\vfill

\item A telescoping series is

\vspace{0.5 in}

\item \textbf{Ex 3:} $\displaystyle \sum_{n=1}^\infty \frac{1}{n(n+1)}$

\vfill

\newpage
\item \textbf{(!!!)} \quad \hspace{.7in} \quad \quad $\displaystyle \sum_{n=1}^\infty  \frac{1}{n} $
\vfill

\item For each series below, determine whether the series converges or diverges. If it converges, determine its sum. State the technique you are using.	\begin{enumerate}
	\item $\displaystyle \sum_{n=1}^\infty \left(\frac{2}{3} \right)^n$
	\vfill
	\item $\displaystyle \sum_{n=1}^\infty 10\left(\frac{-3}{5} \right)^n$
	\vfill
	\item $\displaystyle \sum_{n=1}^\infty  (e^{2/n} - e^{2/(n+1)} )$
	\vfill
	\item $\displaystyle \sum_{n=1}^\infty \left[ \left(\frac{2}{3} \right)^{n} + 10\left(\frac{-3}{5} \right)^n \right] $
	\vspace{.5in}
	\item $\displaystyle \sum_{n=1}^\infty  \frac{\sin(\pi n /2)}{5}$
	\vspace{.5in}
	
	\end{enumerate}

\end{enumerate}
\end{document}

