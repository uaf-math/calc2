
\documentclass[11pt,fleqn]{article} 
\usepackage[margin=0.8in, head=0.8in]{geometry} 
\usepackage{amsmath, amssymb, amsthm}
\usepackage{fancyhdr} 
\usepackage{palatino, url, multicol}
\usepackage{graphicx, pgfplots} 
\usepackage[all]{xy}
\usepackage{polynom} 
%\usepackage{pdfsync} %% I don't know why this messes up tabular column widths
\usepackage{enumerate}
\usepackage{framed}
\usepackage{setspace}
\usepackage{array,tikz}

\pgfplotsset{compat=1.6}

\pgfplotsset{soldot/.style={color=black,only marks,mark=*}} \pgfplotsset{holdot/.style={color=black,fill=white,only marks,mark=*}}


\pagestyle{fancy} 
\lfoot{}
\rfoot{\S 2.2}

\begin{document}
\renewcommand{\headrulewidth}{0pt}
\newcommand{\blank}[1]{\rule{#1}{0.75pt}}
\newcommand{\bc}{\begin{center}}
\newcommand{\ec}{\end{center}}
\renewcommand{\d}{\displaystyle}

\vspace*{-0.7in}

%%%%%%%%%intro page
\begin{center}
  \large
  \sc{Section 2.3: Volumes of Revolution using Cylindrical Shells}\\
  Day 2
\end{center}

\begin{enumerate}
\item In the space below, write the formula for the Cylindrical Shells Method with accompanying formulas. Assume we are integrating with respect to $x.$\\
\vspace{1in}

\item Sketch the region $R$ above the $x$-axis that is bounded by $y=\sqrt{x+2}$ and $y=x$. We want to determine the volume of the solid obtained by rotating $R$ about the $x$-axis. \\

\vspace{1in}


\begin{enumerate}
\item Set up the integral(s) for the volume assuming you are using the Disk/Washer Method. 

\vfill

\item Set up the integral(s) for the volume assuming you are using the Shell Method. 

\vfill


\end{enumerate}
\end{enumerate}
\end{document}

