
\documentclass[11pt,fleqn]{article} 
\usepackage[margin=0.8in, head=0.8in]{geometry} 
\usepackage{amsmath, amssymb, amsthm}
\usepackage{fancyhdr} 
\usepackage{palatino, url, multicol}
\usepackage{graphicx, pgfplots} 
\usepackage[all]{xy}
\usepackage{polynom} 
%\usepackage{pdfsync} %% I don't know why this messes up tabular column widths
\usepackage{enumerate}
\usepackage{framed}
\usepackage{setspace}
\usepackage{array,tikz}

\pgfplotsset{compat=1.6}

\pgfplotsset{soldot/.style={color=black,only marks,mark=*}} \pgfplotsset{holdot/.style={color=black,fill=white,only marks,mark=*}}


\pagestyle{fancy} 
\lfoot{}
\rfoot{\S 2.5}

\begin{document}
\renewcommand{\headrulewidth}{0pt}
\newcommand{\blank}[1]{\rule{#1}{0.75pt}}
\newcommand{\bc}{\begin{center}}
\newcommand{\ec}{\end{center}}
\renewcommand{\d}{\displaystyle}

\vspace*{-0.7in}

%%%%%%%%%intro page
\begin{center}
  \large
  \sc{Section 2.5: Work and Mass (Extra)}\\
   
\end{center}

\begin{enumerate}
\item Recall how we calculated work given both (a) a constant force and (b) a variable force. Recall units.

\vspace{1in}

\item A rectangular fuel oil tank has dimensions $1\: \text{m} \times 1 \: \text{m}$ on the base and is 3 m in height. Assume the depth of the oil in the tank is 2 m. Calculate how much work is required to pump all the oil out of the top of the tank.\\

 (Facts to use: No. 2 fuel oil is roughly 900 kg/$\text{m}^3$. Hence, So the weight (force) density at sea level on earth, of heating oil, is $(9.81\: \text{m}/\text{s}^2) \cdot (900 \:\text{kg}/\text{m}^3) = 8829 \:\text{N}/\text{m}^3.$ This means that a cubic meter of oil on a scale would push down 8829 N, compared to 1 kg of something pushing 9.81 N.)
 \vfill

\end{enumerate}
\end{document}

