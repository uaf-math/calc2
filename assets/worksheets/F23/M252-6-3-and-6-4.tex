\documentclass[11pt]{amsart}
%prepared in AMSLaTeX, under LaTeX2e
\addtolength{\oddsidemargin}{-1.2in} 
\addtolength{\evensidemargin}{-1.2in}
\addtolength{\topmargin}{-.9in}
\addtolength{\textwidth}{2.0in}
\addtolength{\textheight}{1.5in}

\renewcommand{\baselinestretch}{1.05}

\usepackage{verbatim} % for "comment" environment

\usepackage{palatino}

\usepackage[final]{graphicx}

\usepackage{tikz}
\usetikzlibrary{positioning}

\usepackage{enumitem,xspace,fancyvrb}

\newtheorem*{thm}{Theorem}
\newtheorem*{defn}{Definition}
\newtheorem*{example}{Example}
\newtheorem*{problem}{Problem}
\newtheorem*{remark}{Remark}

\DefineVerbatimEnvironment{mVerb}{Verbatim}{numbersep=2mm,frame=lines,framerule=0.1mm,framesep=2mm,xleftmargin=4mm,fontsize=\footnotesize}

% macros
\usepackage{amssymb}
\newcommand{\bA}{\mathbf{A}}
\newcommand{\bB}{\mathbf{B}}
\newcommand{\bE}{\mathbf{E}}
\newcommand{\bF}{\mathbf{F}}
\newcommand{\bJ}{\mathbf{J}}

\newcommand{\bb}{\mathbf{b}}
\newcommand{\br}{\mathbf{r}}
\newcommand{\bv}{\mathbf{v}}
\newcommand{\bw}{\mathbf{w}}
\newcommand{\bx}{\mathbf{x}}

\newcommand{\hbi}{\mathbf{\hat i}}
\newcommand{\hbj}{\mathbf{\hat j}}
\newcommand{\hbk}{\mathbf{\hat k}}
\newcommand{\hbn}{\mathbf{\hat n}}
\newcommand{\hbr}{\mathbf{\hat r}}
\newcommand{\hbt}{\mathbf{\hat t}}
\newcommand{\hbx}{\mathbf{\hat x}}
\newcommand{\hby}{\mathbf{\hat y}}
\newcommand{\hbz}{\mathbf{\hat z}}
\newcommand{\hbphi}{\mathbf{\hat \phi}}
\newcommand{\hbtheta}{\mathbf{\hat \theta}}
\newcommand{\complex}{\mathbb{C}}
\newcommand{\ppr}[1]{\frac{\partial #1}{\partial r}}
\newcommand{\ppt}[1]{\frac{\partial #1}{\partial t}}
\newcommand{\ppx}[1]{\frac{\partial #1}{\partial x}}
\newcommand{\ppy}[1]{\frac{\partial #1}{\partial y}}
\newcommand{\ppz}[1]{\frac{\partial #1}{\partial z}}
\newcommand{\pptheta}[1]{\frac{\partial #1}{\partial \theta}}
\newcommand{\ppphi}[1]{\frac{\partial #1}{\partial \phi}}
\newcommand{\Div}{\ensuremath{\nabla\cdot}}
\newcommand{\Curl}{\ensuremath{\nabla\times}}
\newcommand{\curl}[3]{\ensuremath{\begin{vmatrix} \hbi & \hbj & \hbk \\ \partial_x & \partial_y & \partial_z \\ #1 & #2 & #3 \end{vmatrix}}}
\newcommand{\cross}[6]{\ensuremath{\begin{vmatrix} \hbi & \hbj & \hbk \\ #1 & #2 & #3 \\ #4 & #5 & #6 \end{vmatrix}}}
\newcommand{\eps}{\epsilon}
\newcommand{\grad}{\nabla}
\newcommand{\ip}[2]{\ensuremath{\left<#1,#2\right>}}
\newcommand{\lam}{\lambda}
\newcommand{\lap}{\triangle}

\newcommand{\Null}{\operatorname{null}}
\newcommand{\rank}{\operatorname{rank}}
\newcommand{\range}{\operatorname{range}}
\newcommand{\trace}{\operatorname{tr}}

\newcommand{\RR}{\mathbb{R}}
\newcommand{\ZZ}{\mathbb{Z}}

\newcommand{\prob}[1]{\bigskip\noindent\textbf{#1.}\quad }
\newcommand{\exer}[2]{\prob{Exercise #2 on page #1}}
\newcommand{\exerpages}[2]{\prob{Exercise #2 on pages #1}}

\newcommand{\pts}[1]{(\emph{#1 pts}) }
\newcommand{\epart}[1]{\bigskip\noindent\textbf{(#1)}\quad }
\newcommand{\ppart}[1]{\,\textbf{(#1)}\quad }

\newcommand{\Matlab}{\textsc{Matlab}\xspace}

\newcommand{\ds}{\displaystyle}

\begin{document}

%%%%%%%%%intro page
\begin{center}
  \large
  \sc{Section 6.3: Remainders, Section 6.4: Intro}\\
   
\end{center}

\begin{enumerate}
\item Review from Friday (You should do this from memory OR from your notes from Friday.)
	\begin{enumerate}
	\item If $f(x)$ has derivatives of all orders at $x=a,$ then the \textbf{Taylor series} for $f(x)$ at $x=a$ is\\

	\vspace{1in}

	\item Write the Taylor series for $y=\ln(x)$ at $x=1$ and plug $x=2$ into that formula.

	\vspace{1in}

	\item Write the Taylor series for $f(x)=\cos(x)$ at $a=\pi/2$ and write $p_1(x)$ and $p_3(x)$, the first and third Taylor polynomials for $f(x).$
	
	\vspace{1in}
	\end{enumerate}
\item Using the definition, find the Taylor series for $g(x)=e^x$ centered at $a=0.$ 

\vspace{1.5in}

\item A careful look at \textbf{remainders}, $R_n$, of Taylor series.
\vfill
\newpage

\item Write $p_2(x),$ the 2nd Taylor polynomial for $g(x)=e^x$ centered at $x=0$ and use it to estimate $e.$ Estimate $|R_2|$ on the interval of the $x$-axis between the center ($a=0$) and where we are estimating ($x=1$).
\vfill

\item Use the Taylor series from (2) on this sheet to find (quickly) a Taylor series for $g(x)=e^{x/2}$ and $h(x)=e^{x^2}.$

\vfill
\item Observe that the Taylor series for $h(x)$ allows us to solve a very hard problem!
\vfill
	



\end{enumerate}

\end{document}
