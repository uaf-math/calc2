\documentclass[12pt]{article}
\usepackage[top=1in, bottom=1in, left=.75in, right=.75in]{geometry}
\usepackage{amsmath}
\usepackage{fancyhdr}
\usepackage{graphicx}
\usepackage{txfonts}
\usepackage{multicol,coordsys}
\usepackage[scaled=0.86]{helvet}
\renewcommand{\emph}[1]{\textsf{\textbf{#1}}}
\usepackage{anyfontsize}
\usepackage[shortlabels]{enumitem}
% \usepackage{times}
% \usepackage[lf]{MinionPro}
\usepackage{tikz,pgfplots,mathrsfs}
%\def\degC{{}^\circ{\rm C}}
\def\ra{\rightarrow}
\usetikzlibrary{calc, backgrounds}
\pgfplotsset{compat = newest}
\newcommand{\blank}[1]{\rule{#1}{0.75pt}}

\pgfplotsset{my style/.append style={axis x line=middle, axis y line=
middle, xlabel={$x$}, ylabel={$y$},axis equal}}

%yticklabels={,,} , xticklabels={,,}

% \setmainfont{Times}
% \def\sansfont{Lucida Grande Bold}
\parindent 0pt
\parskip 4pt
\pagestyle{fancy}
\fancyfoot[C]{\emph{\thepage}}
\fancyhead[L]{\ifnum \value{page} > 1\relax\emph{Math 252: Midterm Exam 1}\fi}
\fancyhead[R]{\ifnum \value{page} > 1\relax\emph{Fall 2023}\fi}
\headheight 15pt
\renewcommand{\headrulewidth}{0pt}
\renewcommand{\footrulewidth}{0pt}
\let\ds\displaystyle
\def\continued{{\emph {Continued....}}}
\def\continuing{{\emph {Problem \arabic{probcount} continued....}}\par\vskip 4pt}


\newcounter{probcount}
\newcounter{subprobcount}
\newcommand{\thesubproblem}{\emph{\alph{subprobcount}.}}
\def\problem#1{\setcounter{subprobcount}{0}%
\addtocounter{probcount}{1}{\emph{\arabic{probcount}.\hskip 1em(#1)}}\par}
\def\subproblem#1{\par\hangindent=1em\hangafter=0{%
\addtocounter{subprobcount}{1}\thesubproblem\emph{#1}\hskip 1em}}
\def\probskip{\vskip 10pt}
\def\medprobskip{\vskip 2in}
\def\subprobskip{\vskip 45pt}
\def\bigprobskip{\vskip 4in}

\begin{document}
{\emph{\fontsize{26}{28}\selectfont Math F252\hfill
{\fontsize{32}{36}\selectfont Midterm II}
\hfill Fall 2023}}
\vskip 2cm
\strut\vtop{\halign{\emph#\hskip 0.5em\hfil&#\hbox to 2in{\hrulefill}\cr
\emph{\fontsize{18}{22}\selectfont Name:}&\cr
\noalign{\vskip 10pt}}}
%%\emph{\fontsize{18}{22}\selectfont Student Id:}&\cr
%%\noalign{\vskip 10pt}
%%\emph{\fontsize{18}{22}\selectfont Calculator Model:}&\cr
%}}
%\hfill
%\vtop{\halign{\emph{\fontsize{18}{22}\selectfont #}\hfil& \emph{\fontsize{18}{22}\selectfont\hskip 0.5ex $\square$ #}\hfil\cr
%Section: & 001 (Jill Faudree)\cr
%\noalign{\vskip 4pt}
%         & 002 (James Gossell)\cr
%\noalign{\vskip 4pt}
%         & 005 (James Gossell)\cr}}

\vfill
{\fontsize{18}{22}\selectfont\emph{Rules:}}

You have {90 minutes} to complete this midterm.  

Partial credit will be awarded, but you must show your work.

\textcolor{red}{You may have a single handwritten $3 \times 5$ notecard.}

Calculators are not allowed. 

Place a box around your  \fbox{FINAL ANSWER} to each question where appropriate.

%If you need extra space, you can use the back sides of the pages.
%Please make it obvious  when you have done so.

Turn off anything that might go beep during the exam.

Good luck!
\vfill
\def\emptybox{\hbox to 2em{\vrule height 16pt depth 8pt width 0pt\hfil}}
\def\tline{\noalign{\hrule}}
\centerline{\vbox{\offinterlineskip
{
\bf\sf\fontsize{18pt}{22pt}\selectfont
\hrule
\halign{
\vrule#&\strut\quad\hfil#\hfil\quad&\vrule#&\quad\hfil#\hfil\quad
&\vrule#&\quad\hfil#\hfil\quad&\vrule#\cr
height 3pt&\omit&&\omit&&\omit&\cr
&Problem&&Possible&&Score&\cr\tline
height 3pt&\omit&&\omit&&\omit&\cr
&1&&10&&\emptybox&\cr\tline
&2&&10&&\emptybox&\cr\tline
&3&&20&&\emptybox&\cr\tline
&4&&12&&\emptybox&\cr\tline
&5&&8&&\emptybox&\cr\tline
&6&&30&&\emptybox&\cr\tline
&7&&10&&\emptybox&\cr\tline
&Extra Credit&&5&&\emptybox&\cr\tline
&Total&&100&&\emptybox&\cr
}\hrule}}}

\newpage
\begin{enumerate}
%%% improper integrals
\item (10 points) Compute and simplify the improper integrals, or show that they diverge. Use correct limit notation.
	\begin{enumerate}
	\item $\ds \int_0^1 \frac{dx}{x^4}$
	\vfill
	\item $\ds \int_2^\infty 3x^2e^{-x^3}\: dx$
	\vfill
	\end{enumerate}

\newpage
%%% Sequence of partial sums
\item Consider the infinite series $1-\frac{1}{1\cdot 2}+\frac{1}{1\cdot 2 \cdot 3}-\frac{1}{1\cdot 2 \cdot 3\cdot 4}+\frac{1}{1\cdot 2 \cdot 3\cdot 4\cdot 5}-\frac{1}{1\cdot 2 \cdot 3\cdot 4\cdot 5\cdot 6}+ \cdots.$
	\begin{enumerate}
	\item (5 points) Write the series using sigma or summation notation. (That is, write the series using $\ds \sum $ notation.)
	\vfill
	\item (5 points) Compute and simplify $S_4,$ the partial sum of the first four terms.
	\vfill
	\item (5 points) Does the series converge absolutely, conditionally or neither (diverge)? Show your work and circle one answer. 
	\vfill
\begin{tabular}{ccc}
\quad converges\quad &\quad converges\quad &\quad diverges\quad \\
\quad absolutely\quad&\quad conditionally\quad&\\
\end{tabular}
	\end{enumerate}
\newpage
\item (5 points) Determine if the series $\ds \sum_{n=2}^\infty \frac{1}{n \ln(n)}$ converges or diverges. (Hint: Try the Integral Test.)
\vfill

\item Use $\frac{1}{1-x}=\sum_{n=0}^\infty x^n$ to find power series representations for each function below centered at $a=0.$ 				\begin{enumerate}
	\item (5 points) $g(x)=\frac{2}{3+x}$  
	\vfill
	\item (7 points) $h(x)=\ln(1-x)$
	\vfill
	\end{enumerate}
\item Find the interval of convergence of the following power series.
	\begin{enumerate}
	\item (5 points) $\ds \sum_{n=0}^\infty \frac{x^{2n}}{(n+1)!}$
	\vfill
	\item (7 points) $\ds \sum_{n=1}^\infty \frac{(x-1)^n}{2^n\sqrt{n}}$
	\vfill
	\end{enumerate}
\end{enumerate}
\newpage
%%%
\textbf{Extra Credit} (5 points) 
\end{document}
