\documentclass[addpoints,12pt]{exam}
\pointsinmargin
\marginpointname{ pts}
\pagestyle{headandfoot}
\runningheadrule
\firstpageheadrule
\firstpageheader{Math F252X-901}{Exam 1, Page \thepage\ of \numpages}{Fall 2024}
\runningheader{Math F252X-901}
{Exam 1, Page \thepage\ of \numpages}
{Fall 2024}
\firstpagefooter{}{}{}
\runningfooter{}{}{}

\usepackage{array}
\usepackage{amsmath}
\usepackage{multicol}
\usepackage[margin=0.75in]{geometry}
\usepackage{tikz,enumerate}
\usetikzlibrary{calc,trees,positioning,arrows,fit,shapes,calc}
\usepackage{pgfplots}
\pgfplotsset{width=9cm,compat=1.9}
\tikzset{
  jumpdot/.style={mark=*,solid},
  excl/.append style={jumpdot,fill=white},
  incl/.append style={jumpdot,fill=blue},
}
\renewcommand{\arraystretch}{2.25}
\newcommand{\f}[2]{\displaystyle\frac{#1}{#2}}

\newcommand {\ds}{\displaystyle}
\newcommand{\red}[1]{\textcolor{red}{#1}}

\begin{document}
	
\begin{coverpages}
  \begin{center}
    \begin{center}
      \Large{\textbf{Math F252X-901 - Fall 2024}}\\
      \Large{\textbf{Exam 1}}\\
      \vspace{0.5in}
      \normalsize{\begin{itemize}
        \item No outside materials (e.g. books, notes, calculators,
          other electronic devices).
        \item SHOW ALL WORK. Credit may not be given for answers
          without sufficient work.
        \item Illegible work will not be graded.
        \end{itemize}}
    \end{center}
    \vspace{1in}
		\textbf{Print Name:}\underline{\hspace{4in}}\\
    \vspace{0.5in}
    \gradetable[v][pages]
  \end{center}
\end{coverpages}

\begin{questions}

\question[8] Find the area of the region in the first quadrant bounded
  by the curves $y=x^2+2x$ and $x^3$.\\

\vfill
\vfill

\question[6] Completely set up, but do not evaluate, a definite
  integral which gives the \textbf{arc length} of the curve
  $y=\sqrt{1-x^2}$ from $x=0$ to $x=1$.\\
 
 \vfill
 
 \newpage
 
\question [16] Let $R$ be the region bounded by the curves
  $y=\sqrt{x}$, $x=0$, and $y=2$.
  \begin{parts}
  \part Sketch the region $R$. (\emph{Hint.  Double-check this part
      before proceeding to part (b).})\\
    \begin{center}
      \begin{tikzpicture}[scale=0.3]
        \draw[->, ultra thick] (-2,0) -- (20,0);
        \draw[->, ultra thick] (0,-2) -- (0,10);
        \node at (20,-.5){$x$};
        \node at (-0.5,10){$y$};
      \end{tikzpicture}
    \end{center}
    \vspace{.2in}
  \part Use the \textbf{slicing (disks/washers)} method to completely
    set up, but not evaluate, a definite integral for the volume of the
    solid of revolution formed by rotating $R$ around \textbf{the
      $x$-axis}.
  \vfill
  \part Use the \textbf{shell} method to completely set up, but not
    evaluate, a definite integral for the volume of the same solid of
    revolution as in part \textbf{(b)}.
  \vfill
  \part \textbf{Evaluate one} of the integrals in parts \textbf{(b)}
    or \textbf{(c)} to find the exact volume.
  \vfill
  \vfill
\end{parts}

\newpage

\question[6] Completely set up, but do not evaluate, a definite
  integral for the \textbf{surface area} of the surface created when the
  curve $\ds y=\sin(x)$ on the interval $x=0$ to $x=\pi$ is rotated
  around \textbf{the $x$-axis}.\\

\vfill

\question[10] It takes a force of $4$ Newtons to hold a spring $3$
  centimeters from its equilibrium.
  \begin{parts}
  \part What is the spring constant $k$ in Hooke's Law (i.e.~$F=kx$)?
  \vfill
  \part How much \textbf{work} is done to compress the spring $6$
    centimeters from its equilibrium? Simplify your answer and include
    units.
  \vfill
  \vfill
  \end{parts}

\vfill

\newpage

\question[10] Compute the center of mass $(\bar{x},\bar{y})$ for the
  region in the first quadrant bounded by $y=0$, $x=1$ and $y=2x^2$.\\
 

  \newpage

\question Compute the following integrals and antiderivatives.
  \begin{parts}
  \part[6] $\ds \int ( 2^x + \frac{5}{x} + e^{-3} ) \, dx$
    \vfill
  \part[6] $\ds \int_0^{\pi/2} x\sin(x) \, dx$
    \vfill
  \part[6] $\ds \int x^2e^{(3x)} \, dx$
    \vfill

    \newpage
    
  \part[6] $\ds \int \sin^2(x) \cos^5(x) \, dx$
    \vfill
    
  \part[6] $\ds \int \cos(6x) \sin(2x) \, dx$
    \vfill
    
  \part[7] $\ds \int \frac{dx}{x^2\sqrt{1-x^2}}$
    \vfill

    \newpage

  \part[7] $\ds \int \frac{x}{x^2 -x -6} \, dx$
    \vfill
  \end{parts}

  BONUS (5 points): $\ds \int \frac{x^2}{4+9x^2} \, dx$
  \vfill
  
\end{questions}

\hrulefill

You may find the following trigonometric formulas useful.
\begin{align*}
\sin(\alpha \pm \beta) &= \sin \alpha \cos \beta \pm \cos \alpha \sin \beta \qquad &
    \sin(ax) \sin(bx) &= \frac{1}{2} \cos((a-b)x) - \frac{1}{2} \cos((a+b)x) \\
\cos(\alpha \pm \beta) &= \cos \alpha \cos \beta \mp \sin \alpha \sin \beta \qquad &
    \sin(ax) \cos(bx) &= \frac{1}{2} \sin((a-b)x) + \frac{1}{2} \sin((a+b)x) \\
 && \cos(ax) \cos(bx) &= \frac{1}{2} \cos((a-b)x) + \frac{1}{2} \cos((a+b)x)
\end{align*}

\end{document}


%%% Local Variables:
%%% mode: LaTeX
%%% TeX-master: t
%%% End:
