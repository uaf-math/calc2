\documentclass[addpoints,12pt]{exam}
\pointsinmargin
\marginpointname{ pts}
\pagestyle{headandfoot}
\runningheadrule
\firstpageheadrule
\firstpageheader{Math F252X-901}{Exam 2, Page \thepage\ of \numpages}{Fall 2024}
\runningheader{Math F252X-901}
{Exam 2, Page \thepage\ of \numpages}
{Fall 2024}
\firstpagefooter{}{}{}
\runningfooter{}{}{}

\usepackage{array}
\usepackage{amsmath}
\usepackage{multicol}
\usepackage[margin=0.75in]{geometry}
\usepackage{tikz,enumerate}
\usetikzlibrary{calc,trees,positioning,arrows,fit,shapes,calc}
\usepackage{pgfplots}
\pgfplotsset{width=9cm,compat=1.9}
\tikzset{
  jumpdot/.style={mark=*,solid},
  excl/.append style={jumpdot,fill=white},
  incl/.append style={jumpdot,fill=blue},
}
\renewcommand{\arraystretch}{2.25}
\newcommand{\f}[2]{\displaystyle\frac{#1}{#2}}

\newcommand {\ds}{\displaystyle}
\newcommand{\red}[1]{\textcolor{red}{#1}}

\begin{document}
	
\begin{coverpages}
  \begin{center}
    \begin{center}
      \Large{\textbf{Math F252X-901 - Fall 2024}}\\
      \Large{\textbf{Exam 2}}\\
      \vspace{0.5in}
      \normalsize{\begin{itemize}
        \item No outside materials (e.g. books, notes, calculators,
          other electronic devices).
        \item SHOW ALL WORK. Credit may not be given for answers
          without sufficient work. Cite any tests/theorems you use.
        \item Illegible work will not be graded.
        \end{itemize}}
    \end{center}
    \vspace{1in}
		\textbf{Print Name:}\underline{\hspace{4in}}\\
    \vspace{0.5in}
    \gradetable[v][pages]
  \end{center}
\end{coverpages}

\begin{questions}

\question[6] Evaluate the integral $\ds \int_2^{\infty} \frac{1}{x
    (\ln(x))^2} \, dx$.
  \vfill

\question[6] Evaluate the integral $\ds \int_0^3 \frac{1}{x-1} \, dx$.
  \vfill

\question[6] Find a closed form (explicit formula) for the sequence
  given by the recurrence relation $\ds a_1 = 6000$, $a_{n+1} =
  \frac{a_n}{10}$.
  \vfill

  \newpage

\question[4] Find the first three partial sums for the series $\ds
  \sum_{k=1}^{\infty} (-1)^k(2^{k-1})$.
  \vspace{0.2cm}
  
  $S_1 =$ \underline{\hspace{4cm}}
  \vspace{0.5cm}
  
  $S_2 =$ \underline{\hspace{4cm}}
  \vspace{0.5cm}
  
  $S_3 =$ \underline{\hspace{4cm}}
  \vspace{0.5cm}

\question Determine whether the following series converge or
  diverge. If a series converges, find its sum/limit.
  \begin{parts}
  \part[6] $\ds \sum_{n=1}^{\infty} 3^{1/n} - 3^{1/(n+1)}$
    \vfill
    
  \part[6] $\ds \sum_{n=1}^{\infty} \frac{5}{2^n}$
    \vfill
  \end{parts}

  \newpage

\question Determine whether the following series coverge or diverge.
  \begin{parts}
  \part[6] $\ds \sum_{k=2}^{\infty} \frac{1}{k (\ln(k))^2}$
    %% integral test using #100%
    \vfill
    
  \part[6] $\ds \sum_{k=5}^{\infty} \frac{2k}{k^2 -4k}$
    %% limit comparison test with harmonic series
    \vfill

    \newpage
    
  \part[6] $\ds \sum_{k=50}^{\infty} \frac{1}{e^{1/k}}$
    %% divergence test
    \vfill
    
  \part[6] $\ds \sum_{k=1}^{\infty} \frac{k}{2^k}$
    %% root test
    \vfill

    \newpage
    
  \part[6] $\ds \sum_{k=0}^{\infty} \frac{5^k}{k!}$
    %% ratio test
    \vfill
    
  \end{parts}

\question Determine whether the following series converge
  conditionally, converge absolutely, or diverge.
  \begin{parts}
  \part[6] $\ds \sum_{k=0}^{\infty} \frac{(-5)^k}{k!}$
    %% alt series test + previous problem
    \vfill

    \newpage
    
  \part[6] $\ds \sum_{k=0}^{\infty} \frac{(-1)^{k+1}}{\sqrt{k}}$
    %% abs value is geometric with |r|<1
    \vfill

  \end{parts}

\question[6] Find the center, interval, and radius of convergence for
  the power series $\ds \sum_{n=1}^{\infty} \frac{nx^n}{2^n}$.
  %% ratio test
  \vfill
  
  center:\underline{\hspace{3cm}} \hfill radius:\underline{\hspace{3cm}} \hfill interval:\underline{\hspace{3cm}}

  \newpage
  
\question Find power series representations for the following
  functions. State the
  radius of convergence for each. (Recall that $\ds
  \sum_{k=0}^{\infty} x^k = \frac{1}{1-x}$.)
  \begin{parts}
  \part[6] $\ds f(x) = \frac{1}{1+2x}$
    \vfill
    
  \part[6] $\ds g(x) = \frac{2}{(1+2x)^2}$.
    \vfill
  \end{parts}

  \newpage

\question[6] Find the Maclaurin series (Taylor series centered at 0)
  for the function $f(x) = \sin(x)$.
  \vfill
  
\end{questions}

BONUS: (6 points) Determine whether the series $\ds
\sum_{k=1}^{\infty} \log\left( \frac{n}{n+1} \right)$ converges or
diverges.
%% use properties of logarithms to construct a telescoping series
\vfill


\hrulefill

% You may find the following trigonometric formulas useful.
% \begin{align*}
% \sin(\alpha \pm \beta) &= \sin \alpha \cos \beta \pm \cos \alpha \sin \beta \qquad &
%     \sin(ax) \sin(bx) &= \frac{1}{2} \cos((a-b)x) - \frac{1}{2} \cos((a+b)x) \\
% \cos(\alpha \pm \beta) &= \cos \alpha \cos \beta \mp \sin \alpha \sin \beta \qquad &
%     \sin(ax) \cos(bx) &= \frac{1}{2} \sin((a-b)x) + \frac{1}{2} \sin((a+b)x) \\
%  && \cos(ax) \cos(bx) &= \frac{1}{2} \cos((a-b)x) + \frac{1}{2} \cos((a+b)x)
% \end{align*}

\end{document}


%%% Local Variables:
%%% mode: LaTeX
%%% TeX-master: t
%%% End:
