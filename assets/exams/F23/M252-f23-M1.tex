\documentclass[12pt]{article}
\usepackage[top=1in, bottom=1in, left=.75in, right=.75in]{geometry}
\usepackage{amsmath}
\usepackage{fancyhdr}
\usepackage{graphicx}
\usepackage{txfonts}
\usepackage{multicol,coordsys}
\usepackage[scaled=0.86]{helvet}
\renewcommand{\emph}[1]{\textsf{\textbf{#1}}}
\usepackage{anyfontsize}
\usepackage[shortlabels]{enumitem}
% \usepackage{times}
% \usepackage[lf]{MinionPro}
\usepackage{tikz,pgfplots,mathrsfs}
%\def\degC{{}^\circ{\rm C}}
\def\ra{\rightarrow}
\usetikzlibrary{calc, backgrounds}
\pgfplotsset{compat = newest}
\newcommand{\blank}[1]{\rule{#1}{0.75pt}}

\pgfplotsset{my style/.append style={axis x line=middle, axis y line=
middle, xlabel={$x$}, ylabel={$y$},axis equal}}

%yticklabels={,,} , xticklabels={,,}

% \setmainfont{Times}
% \def\sansfont{Lucida Grande Bold}
\parindent 0pt
\parskip 4pt
\pagestyle{fancy}
\fancyfoot[C]{\emph{\thepage}}
\fancyhead[L]{\ifnum \value{page} > 1\relax\emph{Math 252: Midterm Exam 1}\fi}
\fancyhead[R]{\ifnum \value{page} > 1\relax\emph{Fall 2023}\fi}
\headheight 15pt
\renewcommand{\headrulewidth}{0pt}
\renewcommand{\footrulewidth}{0pt}
\let\ds\displaystyle
\def\continued{{\emph {Continued....}}}
\def\continuing{{\emph {Problem \arabic{probcount} continued....}}\par\vskip 4pt}


\newcounter{probcount}
\newcounter{subprobcount}
\newcommand{\thesubproblem}{\emph{\alph{subprobcount}.}}
\def\problem#1{\setcounter{subprobcount}{0}%
\addtocounter{probcount}{1}{\emph{\arabic{probcount}.\hskip 1em(#1)}}\par}
\def\subproblem#1{\par\hangindent=1em\hangafter=0{%
\addtocounter{subprobcount}{1}\thesubproblem\emph{#1}\hskip 1em}}
\def\probskip{\vskip 10pt}
\def\medprobskip{\vskip 2in}
\def\subprobskip{\vskip 45pt}
\def\bigprobskip{\vskip 4in}

\begin{document}
{\emph{\fontsize{26}{28}\selectfont Math F252\hfill
{\fontsize{32}{36}\selectfont Midterm I}
\hfill Fall 2023}}
\vskip 2cm
\strut\vtop{\halign{\emph#\hskip 0.5em\hfil&#\hbox to 2in{\hrulefill}\cr
\emph{\fontsize{18}{22}\selectfont Name:}&\cr
\noalign{\vskip 10pt}}}
%%\emph{\fontsize{18}{22}\selectfont Student Id:}&\cr
%%\noalign{\vskip 10pt}
%%\emph{\fontsize{18}{22}\selectfont Calculator Model:}&\cr
%}}
%\hfill
%\vtop{\halign{\emph{\fontsize{18}{22}\selectfont #}\hfil& \emph{\fontsize{18}{22}\selectfont\hskip 0.5ex $\square$ #}\hfil\cr
%Section: & 001 (Jill Faudree)\cr
%\noalign{\vskip 4pt}
%         & 002 (James Gossell)\cr
%\noalign{\vskip 4pt}
%         & 005 (James Gossell)\cr}}

\vfill
{\fontsize{18}{22}\selectfont\emph{Rules:}}

You have {90 minutes} to complete this midterm.  

Partial credit will be awarded, but you must show your work.

You may have a single handwritten $3 \times 5$ notecard.

Calculators are not allowed. 

Place a box around your  \fbox{FINAL ANSWER} to each question where appropriate.

%If you need extra space, you can use the back sides of the pages.
%Please make it obvious  when you have done so.

Turn off anything that might go beep during the exam.

Good luck!
\vfill
\def\emptybox{\hbox to 2em{\vrule height 16pt depth 8pt width 0pt\hfil}}
\def\tline{\noalign{\hrule}}
\centerline{\vbox{\offinterlineskip
{
\bf\sf\fontsize{18pt}{22pt}\selectfont
\hrule
\halign{
\vrule#&\strut\quad\hfil#\hfil\quad&\vrule#&\quad\hfil#\hfil\quad
&\vrule#&\quad\hfil#\hfil\quad&\vrule#\cr
height 3pt&\omit&&\omit&&\omit&\cr
&Problem&&Possible&&Score&\cr\tline
height 3pt&\omit&&\omit&&\omit&\cr
&1&&10&&\emptybox&\cr\tline
&2&&10&&\emptybox&\cr\tline
&3&&20&&\emptybox&\cr\tline
&4&&12&&\emptybox&\cr\tline
&5&&8&&\emptybox&\cr\tline
&6&&30&&\emptybox&\cr\tline
&7&&10&&\emptybox&\cr\tline
&Extra Credit&&5&&\emptybox&\cr\tline
&Total&&100&&\emptybox&\cr
}\hrule}}}

\newpage
\begin{enumerate}
%%% area between curves
\item (10 points) Find the area between the curves {$\displaystyle{y=\frac{10}{x+1}}$} and {$\displaystyle{y=10-2x}$} on the interval $[0,4]$. Your final answer should be a reasonably simplified number. \\

\vfill


%%Arc length
\item (10 points) Find the surface area of the volume generated when the curve $y=\frac{x^3}{3}$ from $x=0$ to $x=2$ is revolved around the $x$-axis. Your final answer should be a reasonably simplified number. \\

\vfill

\newpage

%%%Volumes of Revolution
\item (20 points) Let $R$ be the region in the first quadrant bounded by $y=\sqrt{x}$, $y=0$, and $x=16$. 
\begin{enumerate}[(a)]
	\item Sketch the region $R$ on the given graph.\\
	\begin{center}
        \begin{tikzpicture}[scale=0.3]
        \draw[->, ultra thick] (-2,0) -- (20,0);
        \draw[->, ultra thick] (0,-2) -- (0,10);
        \node at (20,-.5){$x$};
         \node at (-0.5,10){$y$};
         \end{tikzpicture}
        \end{center}
	
\item For each problem below, set up -- but do not evaluate -- a definite integral for the volume of the solid of revolution formed by rotating the region $R$ about the given axis using the given method. \emph{Your final answer should be in a form that is immediately integrable without any additional algebra.}
\begin{enumerate}
	\item Axis of rotation: $x$-axis, Method: disks/washers.\\
	\vfill
	\item Axis of rotation: $x$-axis, Method: shells.	\\
	\vfill
	\item Axis of rotation: $y$-axis, Method: disks/washers.	\\
	\vfill
	\item Axis of rotation: $y$-axis, Method: shells.	\\
	\vfill
	\end{enumerate}
\end{enumerate}

\pagebreak

%%% spring problem
\item (12 points) A spring has a natural length of 30 cm (or $0.3$ m). It takes a force of 40 N to hold the spring at a length of 40 cm (or $0.4$ m). \\
\begin{enumerate}
	\item What is the spring constant $k$ in Hooke's Law?
	\vfill
	\item How much \textbf{work} is done to stretch the spring to a length of 50 cm  (or $0.5$ m)? Simplify your answer and include units.
	\vfill
	\end{enumerate}
\vfill 

%%% center of mass
\item (8 points) Set up but do not evaluate the three integrals needed to compute the center of mass, $(\overline{x},\overline{y}),$ of the region $R$ bounded by $y=0$, $x=0$, $x=2$, and $f(x)=2e^x$ with constant density $\rho=1.$ Then, fill in the blanks at the bottom to show how to compute the values of $(\overline{x},\overline{y}).$ 

\vfill
{\Large{$m=$}}\\

\vfill
{\Large{$M_y=$}}\\

\vfill
{\Large{$M_x=$}}\\

\vfill
{\Large{$\overline{x}=\frac{\framebox(50,30){}}{\framebox(50,30){}} \hspace{1in} \overline{y}=\frac{\framebox(50,30){}}{\framebox(50,30){}}$}}

\pagebreak

%%% three integrals
\item (30 points) Evaluate the following indefinite integrals. Show your work and simplify your answers.
\begin{enumerate}[(a)]
\item $\displaystyle \int x^2e^x \,dx$
\vfill
\item $\displaystyle \int \sin^3 (\theta) \cos^2 (\theta) \,d\theta$
\vfill
\newpage
\item $\displaystyle \int \frac{4-x^2}{x^3+2x}\,dx$
\vfill
\item $\displaystyle \int x\sec^2(x) \,dx$
\vfill
\end{enumerate}

\pagebreak

%%% trig sub
\item (10 points) Evaluate the integral $\displaystyle \int \frac{x^2}{\sqrt{4-x^2}} \: dx$ using trigonometric substitution. You must \emph{fully} simplify your answer.\\

\vfill

\end{enumerate}
\newpage
%%%
\textbf{Extra Credit} (5 points) 
	\begin{enumerate}[(a)]
	\item Use integration by parts to prove the reduction formula $\displaystyle{\int ( \ln x )^n \: dx = x( \ln x )^n - n\int (\ln x ) ^{n-1} \: dx. }$
	\vfill
	\item Use the reduction formula to evaluate $\int (\ln x)^2 \: dx.$
	\vfill
	\end{enumerate}
\vfill 
\newpage

\begin{tabular}{ccc}
$\sin^2(x)=\frac{1}{2} ( 1-\cos(2x))$&&$\sin(ax)\cos(bx)=\frac{1}{2} ( \sin((a-b)x)+\sin((a+b)x))$\\
&&\\
 $\cos^2(x)=\frac{1}{2} ( 1+\cos(2x))$&&$\sin(ax)\sin(bx)=\frac{1}{2} ( \cos((a-b)x)-\cos((a+b)x))$\\
 &&\\
$\sin(2 \theta)=2\sin(\theta) \cos(\theta)$ &&$\cos(ax)\cos(bx)=\frac{1}{2} ( \cos((a-b)x)+\cos((a+b)x))$\\

\end{tabular}
\end{document}
