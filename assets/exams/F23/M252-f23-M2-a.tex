\documentclass[12pt]{article}
\usepackage[top=1in, bottom=1in, left=.75in, right=.75in]{geometry}
\usepackage{amsmath}
\usepackage{fancyhdr}
\usepackage{graphicx}
\usepackage{txfonts}
\usepackage{multicol,coordsys}
\usepackage[scaled=0.86]{helvet}
\renewcommand{\emph}[1]{\textsf{\textbf{#1}}}
\usepackage{anyfontsize}
\usepackage[shortlabels]{enumitem}
% \usepackage{times}
% \usepackage[lf]{MinionPro}
\usepackage{tikz,pgfplots,mathrsfs}
%\def\degC{{}^\circ{\rm C}}
\def\ra{\rightarrow}
\usetikzlibrary{calc, backgrounds}
\pgfplotsset{compat = newest}
\newcommand{\blank}[1]{\rule{#1}{0.75pt}}

\pgfplotsset{my style/.append style={axis x line=middle, axis y line=
middle, xlabel={$x$}, ylabel={$y$},axis equal}}

%yticklabels={,,} , xticklabels={,,}

% \setmainfont{Times}
% \def\sansfont{Lucida Grande Bold}
\parindent 0pt
\parskip 4pt
\pagestyle{fancy}
\fancyfoot[C]{\emph{\thepage}}
\fancyhead[L]{\ifnum \value{page} > 1\relax\emph{Math 252: Midterm Exam 1}\fi}
\fancyhead[R]{\ifnum \value{page} > 1\relax\emph{Fall 2023}\fi}
\headheight 15pt
\renewcommand{\headrulewidth}{0pt}
\renewcommand{\footrulewidth}{0pt}
\let\ds\displaystyle
\def\continued{{\emph {Continued....}}}
\def\continuing{{\emph {Problem \arabic{probcount} continued....}}\par\vskip 4pt}


\newcounter{probcount}
\newcounter{subprobcount}
\newcommand{\thesubproblem}{\emph{\alph{subprobcount}.}}
\def\problem#1{\setcounter{subprobcount}{0}%
\addtocounter{probcount}{1}{\emph{\arabic{probcount}.\hskip 1em(#1)}}\par}
\def\subproblem#1{\par\hangindent=1em\hangafter=0{%
\addtocounter{subprobcount}{1}\thesubproblem\emph{#1}\hskip 1em}}
\def\probskip{\vskip 10pt}
\def\medprobskip{\vskip 2in}
\def\subprobskip{\vskip 45pt}
\def\bigprobskip{\vskip 4in}

\begin{document}
{\emph{\fontsize{26}{28}\selectfont Math F252\hfill
{\fontsize{32}{36}\selectfont Midterm II}
\hfill Fall 2023}}
\vskip 2cm
\strut\vtop{\halign{\emph#\hskip 0.5em\hfil&#\hbox to 2in{\hrulefill}\cr
\emph{\fontsize{18}{22}\selectfont Name:}&\cr
\noalign{\vskip 10pt}}}
%%\emph{\fontsize{18}{22}\selectfont Student Id:}&\cr
%%\noalign{\vskip 10pt}
%%\emph{\fontsize{18}{22}\selectfont Calculator Model:}&\cr
%}}
%\hfill
%\vtop{\halign{\emph{\fontsize{18}{22}\selectfont #}\hfil& \emph{\fontsize{18}{22}\selectfont\hskip 0.5ex $\square$ #}\hfil\cr
%Section: & 001 (Jill Faudree)\cr
%\noalign{\vskip 4pt}
%         & 002 (James Gossell)\cr
%\noalign{\vskip 4pt}
%         & 005 (James Gossell)\cr}}

\vfill
{\fontsize{18}{22}\selectfont\emph{Rules:}}

You have {90 minutes} to complete this midterm.  

Partial credit will be awarded, but you must show your work.

%\textcolor{red}{You may have a single handwritten $3 \times 5$ notecard.}

Calculators, notes and books are not allowed. 

%Place a box around your  \fbox{FINAL ANSWER} to each question where appropriate.

%If you need extra space, you can use the back sides of the pages.
%Please make it obvious  when you have done so.

Turn off anything that might go beep during the exam.

Good luck!
\vfill
\def\emptybox{\hbox to 2em{\vrule height 16pt depth 8pt width 0pt\hfil}}
\def\tline{\noalign{\hrule}}
\centerline{\vbox{\offinterlineskip
{
\bf\sf\fontsize{18pt}{22pt}\selectfont
\hrule
\halign{
\vrule#&\strut\quad\hfil#\hfil\quad&\vrule#&\quad\hfil#\hfil\quad
&\vrule#&\quad\hfil#\hfil\quad&\vrule#\cr
height 3pt&\omit&&\omit&&\omit&\cr
&Problem&&Possible&&Score&\cr\tline
height 3pt&\omit&&\omit&&\omit&\cr
&1&&12&&\emptybox&\cr\tline
&2&&5&&\emptybox&\cr\tline
&3&&9&&\emptybox&\cr\tline
&4&&10&&\emptybox&\cr\tline
&5&&10&&\emptybox&\cr\tline
&6&&24&&\emptybox&\cr\tline
&7&&10&&\emptybox&\cr\tline
&8&&10&&\emptybox&\cr\tline
&9&&10&&\emptybox&\cr\tline
&Extra Credit&&5&&\emptybox&\cr\tline
&Total&&100&&\emptybox&\cr
}\hrule}}}

\newpage
\begin{enumerate}
%%% improper integrals
\item (12 points) Compute and simplify the improper integrals, or show that they diverge. Use correct limit notation.
	\begin{enumerate}
	\item $\ds \int_2^\infty \frac{dx}{x(\ln(x))^2}$
	\vfill
	\item $\ds \int_0^3 \frac{1}{x^{4/3}}\: dx$
	\vfill
	\end{enumerate}

\item (5 points) Does the series $\ds \sum_{n=2}^\infty \frac{1}{n(\ln(n))^2}$  converge or diverge? Show your work including naming any test you use. (Hint: You may use the previous problem though you don't have to. )\\


\vfill

\newpage
%%% Sequence of partial sums
\item (9 points) Consider the infinite series $-\frac{1}{2\cdot 1}+\frac{1}{2\cdot 2}-\frac{1}{2\cdot3}+\frac{1}{2\cdot 4}-\frac{1}{2\cdot 5}+\frac{1}{2 \cdot 6}-\cdots$
	\begin{enumerate}
	\item Write the series using sigma or summation notation. (That is, write the series using $\ds \sum $ notation.)
	\vfill
	\item  Compute and simplify $S_1,$ $S_2,$ and $S_3$ the first three terms in the sequence partial sums of the series.
	\vfill
	\end{enumerate}
\item (10 points) Consider the infinite series $\ds \sum_{n=0}^\infty \frac{(-3)^{n+1}}{10^{n}}.$
	\begin{enumerate}
	\item Explain why the series converges. 
	\vfill
	\item Determine the sum of the series. Write you answer as a simplified fraction.
	\vfill
	\end{enumerate}
\newpage
\item (10 points) Show that the series $\quad \ds \sum_{n=0}^\infty (-1)^{n+1}\frac{1}{\sqrt{2n+1}} \quad$ is conditionally convergent. \\

Note that you must show that the series converges \textbf{and} that it is not absolutely convergent. \\

A complete answer will include (i) the name of the test(s) you are using, (ii) a clear application of the test (or tests), and (iii) an explicit explanation of what conclusion(s) you are drawing.
		
\newpage
\item (24 points) Do the following series converge or diverge? Show your work, including naming any test you use.
	\begin{enumerate}
	\item $\ds \sum_{n=0}^\infty \frac{2n-1}{5n+1}$
	\vfill
%	\item $\ds \sum_{n=0}^\infty \frac{n+2}{6^n}$
%	\vfill
	\item $\ds \sum_{n=1}^\infty \frac{\ln(n)}{n^2} $
	\vfill
	\newpage
	\item $\ds \sum_{n=1}^\infty \frac{\sin(n)}{n^{5/3}}$
	\vfill
	\item $\ds \sum_{n=1}^\infty \frac{(\ln n)^{n}}{n^n} $
	\vfill
	\end{enumerate}
\newpage
\item (10 points) Use $\quad \ds \frac{1}{1-x}=\sum_{n=0}^\infty x^n \quad$ to find power series representations centered at $a=0$ for each function below.  				\begin{enumerate}
	\item $g(x)=\frac{x}{1-3x}$  
	\vspace{3in}
	\item $h(x)=\frac{1}{(1+x)^2}$ (Hint: Differentiate an appropriate function.)
	\vfill
	\end{enumerate}
\newpage
\item (10 points) Write the Taylor series for $y=e^{-2x}$ centered at $a=1.$
\vfill

\item (10 points) Find the interval of convergence of the following power series.
	\begin{enumerate}
	\item $\ds \sum_{n=0}^\infty \frac{(x-3)^{n}}{(n+1)!}$
	\vfill
	\item $\ds \sum_{n=1}^\infty \frac{x^n}{n8^n}$
	\vfill
	\end{enumerate}

\end{enumerate}
\newpage
%%%
\textbf{Extra Credit} (5 points) The Taylor series for $f(x)=\sin(x)$ centered at $a=0$ is $\ds \sin(x)=\sum_{n=0}^\infty \frac{(-1)^n}{(2n+1)!}x^{2n+1}.$
\begin{enumerate}
\item Find $p_3(x)$, the 3rd Taylor polynomial, and use it to estimate $\sin(1).$
\item Show that this estimate is within $0.005$ of the exact value.
\end{enumerate}
\vfill
\end{document}
