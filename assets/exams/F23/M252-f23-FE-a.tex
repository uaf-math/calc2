\documentclass[12pt]{article}
\usepackage[top=1in, bottom=1in, left=.75in, right=.75in]{geometry}
\usepackage{amsmath}
\usepackage{fancyhdr}
\usepackage{graphicx}
\usepackage{txfonts}
\usepackage{multicol,coordsys}
\usepackage[scaled=0.86]{helvet}
\renewcommand{\emph}[1]{\textsf{\textbf{#1}}}
\usepackage{anyfontsize}
\usepackage[shortlabels]{enumitem}
% \usepackage{times}
% \usepackage[lf]{MinionPro}
\usepackage{tikz,pgfplots,mathrsfs}
%\def\degC{{}^\circ{\rm C}}
\def\ra{\rightarrow}
\usetikzlibrary{calc, backgrounds}
\pgfplotsset{compat = newest}
\newcommand{\blank}[1]{\rule{#1}{0.75pt}}

\pgfplotsset{my style/.append style={axis x line=middle, axis y line=
middle, xlabel={$x$}, ylabel={$y$},axis equal}}

%yticklabels={,,} , xticklabels={,,}

% \setmainfont{Times}
% \def\sansfont{Lucida Grande Bold}
\parindent 0pt
\parskip 4pt
\pagestyle{fancy}
\fancyfoot[C]{\emph{\thepage}}
\fancyhead[L]{\ifnum \value{page} > 1\relax\emph{Math 252: Final Exam}\fi}
\fancyhead[R]{\ifnum \value{page} > 1\relax\emph{Fall 2023}\fi}
\headheight 15pt
\renewcommand{\headrulewidth}{0pt}
\renewcommand{\footrulewidth}{0pt}
\let\ds\displaystyle
\def\continued{{\emph {Continued....}}}
\def\continuing{{\emph {Problem \arabic{probcount} continued....}}\par\vskip 4pt}


\newcounter{probcount}
\newcounter{subprobcount}
\newcommand{\thesubproblem}{\emph{\alph{subprobcount}.}}
\def\problem#1{\setcounter{subprobcount}{0}%
\addtocounter{probcount}{1}{\emph{\arabic{probcount}.\hskip 1em(#1)}}\par}
\def\subproblem#1{\par\hangindent=1em\hangafter=0{%
\addtocounter{subprobcount}{1}\thesubproblem\emph{#1}\hskip 1em}}
\def\probskip{\vskip 10pt}
\def\medprobskip{\vskip 2in}
\def\subprobskip{\vskip 45pt}
\def\bigprobskip{\vskip 4in}

\begin{document}
{\emph{\fontsize{26}{28}\selectfont Math F252\hfill
{\fontsize{32}{36}\selectfont Final Exam}
\hfill Fall 2023}}
\vskip 2cm
\strut\vtop{\halign{\emph#\hskip 0.5em\hfil&#\hbox to 2in{\hrulefill}\cr
\emph{\fontsize{18}{22}\selectfont Name:}&\cr
\noalign{\vskip 10pt}}}
%%\emph{\fontsize{18}{22}\selectfont Student Id:}&\cr
%%\noalign{\vskip 10pt}
%%\emph{\fontsize{18}{22}\selectfont Calculator Model:}&\cr
%}}
%\hfill
%\vtop{\halign{\emph{\fontsize{18}{22}\selectfont #}\hfil& \emph{\fontsize{18}{22}\selectfont\hskip 0.5ex $\square$ #}\hfil\cr
%Section: & 001 (Jill Faudree)\cr
%\noalign{\vskip 4pt}
%         & 002 (James Gossell)\cr
%\noalign{\vskip 4pt}
%         & 005 (James Gossell)\cr}}

\vfill
{\fontsize{18}{22}\selectfont\emph{Rules:}}

You have 2 hours to complete this midterm.  

Partial credit will be awarded, but you must show your work.

{You may have a single sheet of paper written on the front only.}

Calculators and books are not allowed. 

%Place a box around your  \fbox{FINAL ANSWER} to each question where appropriate.

%If you need extra space, you can use the back sides of the pages.
%Please make it obvious  when you have done so.

Turn off anything that might go beep during the exam.

Good luck!
\vfill
\def\emptybox{\hbox to 2em{\vrule height 16pt depth 8pt width 0pt\hfil}}
\def\tline{\noalign{\hrule}}
\centerline{\vbox{\offinterlineskip
{
\bf\sf\fontsize{18pt}{22pt}\selectfont
\hrule
\halign{
\vrule#&\strut\quad\hfil#\hfil\quad&\vrule#&\quad\hfil#\hfil\quad
&\vrule#&\quad\hfil#\hfil\quad&\vrule#\cr
height 3pt&\omit&&\omit&&\omit&\cr
&Problem&&Possible&&Score&\cr\tline
height 3pt&\omit&&\omit&&\omit&\cr
&1&&10&&\emptybox&\cr\tline
&2&&5&&\emptybox&\cr\tline
&3&&5&&\emptybox&\cr\tline
&4&&10&&\emptybox&\cr\tline
&5&&10&&\emptybox&\cr\tline
&6&&10&&\emptybox&\cr\tline
&7&&8&&\emptybox&\cr\tline
&8&&8&&\emptybox&\cr\tline
&9&&6&&\emptybox&\cr\tline
&10&&12&&\emptybox&\cr\tline
&11&&8&&\emptybox&\cr\tline
&12&&8&&\emptybox&\cr\tline
&Extra Credit&&5&&\emptybox&\cr\tline
&Total&&100&&\emptybox&\cr
}\hrule}}}

\newpage
\begin{enumerate}
%Area and Volume Rectangular
\item (10 points) Let $R$ be the region of the plane bounded by $y=3\sqrt{x},$ the $y$-axis, and $y=3.$
	\begin{enumerate}
	\item Sketch the region $R$. Label at least three points on your graph.\\
	
		\begin{tikzpicture}[scale=1.5]
		\draw[ultra thick,<->] (-1,0) -- (4,0);
		\draw[ultra thick,<->] (0,-1) -- (0,2);
		\end{tikzpicture}
	\item Find the area of the region $R.$ Your final answer should be simplified.\\
	\vfill
	\item Find the volume of the solid obtained by rotating the region $R$ about the $x$-axis. Your final answer should be simplified.\\
	\vfill
	\end{enumerate}
	\newpage
%integration techniques (1) (definite)
\item (5 points) A $1$-meter long rod oriented along the x-axis on the interval $[0,1]$ has density $\rho(x)=xe^{3x}$ grams per meter at position $x$ meters. Find the mass of the rod. Include units in your answer.
	\vfill
\item (5 points) Evaluate the definite integral: $\ds \int_0^1 \frac{5x+1}{(x+1)(2x+1)}\: dx$
	\vfill
	\newpage
%integration techniques (2) (indefinite)
\item (10 points) Evaluate the following indefinite integrals.
	\begin{enumerate}
	\item $\ds \int \sec^4(\theta)\: d\theta$
	\vfill
	\item $\ds \int \frac{x^2}{\sqrt{4-x^2}}\: dx$
	\vfill
	\end{enumerate}
\newpage
%series techniques (1)
\item (10 points) Determine whether the series is convergent or divergent. Note that to earn full credit, you work must include the name of the test being applied, a clear application of the test, and a conclusion. 
	\begin{enumerate}
	\item $\ds \sum_{n=0}^\infty \frac{n^2+1}{n^3+1}$
	\vfill
	\item $\ds \sum_{n=0}^\infty \frac{(-1)^n}{\sqrt{2n+1}}$
	\vfill
	\end{enumerate}
\newpage
%series techniques (2)
\item (10 points) Determine whether the series is convergent or divergent. Note that to earn full credit, you work must include the name of the test being applied, a clear application of the test, and a conclusion. 
	\begin{enumerate}
	\item $\ds \sum_{n=0}^\infty \ln \left( \frac{2n}{3n+5}\right)$
	\vfill
	\item $\ds 1+e+\frac{e^2}{2!}+\frac{e^3}{3!}+\frac{e^4}{4!}+\cdots$
	\vfill
	\end{enumerate}
\newpage
%Improper integral
\item (8 points) Evaluate the improper integral $\ds \int_2^\infty \frac{dx}{x (\ln(x))^2}$ or demonstrate that it is divergent. Use correct limit notation.
\vfill
%radius/interval of convergence
\item (8 points) Find the interval of convergence for the series $\ds \sum_{n=1}^\infty \frac{(x-2)^n}{n3^n}$
\vfill
\vfill
\newpage
%taylor/maclaurin series + sense-making
\item (6 points) Find the Taylor series for $f(x)=e^{5x}$ centered at $a=-1.$
\vfill
\newpage
% parametric equations
\item (12 points) Answer the questions about the parametric equations $x(t)=t^2$ and $y(t)=t^3+t,$ where $0 \leq t.$
	\begin{enumerate}
	\item Make a rough sketch of the curve defined by the parametric equations. Plot at least 4 points and indicate with arrows the direction in which the curve is traced as $t$ increases.\\
	
\vfill

	\item Write an equation of the line tangent to the curve when $t=1.$
	\vfill
	\item Use the second derivative to demonstrate that the curve is concave up when $t=1.$
	\vfill
	\end{enumerate}
\newpage	
% elementary polar coordinates
\item (8 points) Make a careful and reasonably large sketch of the polar curve $r=1+\cos(\theta).$ To earn full credit, you must label at least $4$ points on your graph.
\vfill
%solution to DE OR area in polar coord
\item (8 points) Find the area inside the polar curve in the previous problem.
\vfill
\end{enumerate}
\newpage
%%%Extra Credit
\textbf{Extra Credit} (5 points) Determine all the values of $p$ for which the series $\ds \sum_{n=0}^\infty \frac{4^{pn}}{3^n}$ converges or explain why it is not possible for the series to converge for any value of $p.$
\end{document}
