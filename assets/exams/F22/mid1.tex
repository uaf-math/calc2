\documentclass[11pt]{amsart}
%\pagestyle{empty} 
\setlength{\topmargin}{-0.5in} % usually -0.25in
\addtolength{\textheight}{1.2in} % usually 1.25in
\addtolength{\oddsidemargin}{-0.95in}
\addtolength{\evensidemargin}{-0.95in}
\addtolength{\textwidth}{1.9in} %\setlength{\parindent}{0pt}

\newcommand{\normalspacing}{\renewcommand{\baselinestretch}{1.1}\tiny\normalsize}
\normalspacing

% macros
\usepackage{amssymb,xspace,alltt,verbatim}
\usepackage[final]{graphicx}
\usepackage[pdftex,colorlinks=true]{hyperref}
\usepackage{fancyvrb}
\usepackage{tikz}

\newtheorem*{lem*}{Lemma}

\newcommand{\bb}{\mathbf{b}}
\newcommand{\bc}{\mathbf{c}}
\newcommand{\bs}{\mathbf{s}}
\newcommand{\bu}{\mathbf{u}}
\newcommand{\bv}{\mathbf{v}}
\newcommand{\bw}{\mathbf{w}}
\newcommand{\bx}{\mathbf{x}}
\newcommand{\by}{\mathbf{y}}

\newcommand{\bbf}{\mathbf{f}}

\newcommand{\CC}{{\mathbb{C}}}
\newcommand{\RR}{{\mathbb{R}}}
\newcommand{\eps}{\epsilon}
\newcommand{\ZZ}{{\mathbb{Z}}}
\newcommand{\ZZn}{{\mathbb{Z}}_n}
\newcommand{\NN}{{\mathbb{N}}}
\newcommand{\ip}[2]{\mathrm{\left<#1,#2\right>}}

\renewcommand{\Re}{\operatorname{Re}}
\renewcommand{\Im}{\operatorname{Im}}

\newcommand{\Log}{\operatorname{Log}}

\newcommand{\grad}{\nabla}

\newcommand{\ds}{\displaystyle}

\newcommand{\Matlab}{\textsc{Matlab}\xspace}
\newcommand{\Octave}{\textsc{Octave}\xspace}
\newcommand{\pylab}{\textsc{pylab}\xspace}

\newcommand{\prob}[1]{\bigskip\noindent\textbf{#1.} }
\newcommand{\pts}[1]{(\emph{#1 pts})}

\newcommand{\probpts}[2]{\prob{#1} \pts{#2} \quad}
\newcommand{\ppartpts}[2]{\textbf{(#1)} \pts{#2} \quad}
\newcommand{\epartpts}[2]{\medskip\noindent \textbf{(#1)} \pts{#2} \quad}


\begin{document}
\hfill \Large Name:\underline{\phantom{Ed Bueler really really long long long name}}
\medskip

\scriptsize \noindent Math 252 Calculus 2 (Bueler) \hfill Thursday, 6 October 2022
\medskip

\LARGE\centerline{\textbf{Midterm Exam 1}}

\smallskip
\begin{quote}
\large
\textbf{No book, notes, electronics, calculator, or internet access.  100 points possible.  70 minutes maximum.}
\end{quote}

\normalsize
\medskip

\thispagestyle{empty}

\probpts{1}{7}  Compute the area between the curves $y=x^2$ and $y=\sqrt{x}$ on the interval $0\le x \le 1$. \quad (\emph{Hint.  Be careful about which curve is above the other.})
\vfill

\probpts{2}{6}  Completely set up, but do not evaluate, a definite integral for the \textbf{length} of the curve $\ds y=\sqrt{x}$ on the interval $x=1$ to $x=4$.
\vfill

\clearpage\newpage
\prob{3}  \ppartpts{a}{4} Sketch the region bounded by the curves $y=e^x$, $x=0$, and $y=e$. \quad (\emph{Hint.  Double-check this part!})

\vfill

\epartpts{b}{4} Use the \textbf{slicing (disks/washers)} method to completely set up, but not evaluate, a definite integral for the volume of the solid of revolution formed by rotating the region in part \textbf{(a)} around \textbf{the $y$-axis}.

\vspace{1.2in}

\epartpts{c}{4} Use the \textbf{shells} method to completely set up, but not evaluate, a definite integral for the volume of the same solid of revolution as in part \textbf{(b)}.

\vspace{1.2in}

\epartpts{d}{4} Evaluate one of the integrals in parts \textbf{(b)} or \textbf{(c)} to find the volume.
\vfill

\clearpage\newpage
\probpts{4}{6}  Completely set up, but do not evaluate, a definite integral for the \textbf{surface area} of the surface created when the curve $\ds y=x^2$ on the interval $x=0$ to $x=1$ is rotated around \textbf{the $x$-axis}.
\vfill

\prob{5} It takes a force of $4$ Newtons to hold a spring $3$ centimeters from its equilibrium.

\epartpts{a}{3} What is the spring constant $k$ in Hooke's Law (i.e.~$F=kx$)?

\vspace{1.5in}

\epartpts{b}{6} How much \textbf{work} is done to compress the spring $6$ centimeters from its equilibrium? Simplify your answer and include units.


\vfill

\clearpage\newpage
\prob{6}  Evaluate and simplify the following indefinite and definite integrals.

\epartpts{a}{6}  $\ds \int_0^2 5^x \,dx = $
\vfill

\epartpts{b}{6}  $\ds \int \cot \theta\,d\theta = $
\vfill

\epartpts{c}{6}  $\ds \int \cos (7t) \sin (7t) \,dt = $
\vfill

\clearpage\newpage
\epartpts{d}{6}  $\ds \int_0^{\pi/2} \sin^3 x \,dx = $
\vfill

\epartpts{e}{6}  $\ds \int x^2 \sin x\,dx = $
\vfill

\clearpage\newpage
\epartpts{f}{6}  $\ds \int \sec x\,dx = $
\vfill

\epartpts{g}{6}  $\ds \int \sin(7x)\cos(3x)\,dx = $
\vfill

\clearpage\newpage
\probpts{7}{8}  Evaluate and simplify the indefinite integral:

\medskip
    $$\int \frac{x^2+x+1}{x^3+x}\,dx = \hspace{5.0in}$$
\vfill

\probpts{8}{8}  Evaluate and \emph{fully} simplify the indefinite integral.

\noindent (\emph{Hint.} $(\tan \theta)' = \sec^2\theta$ and $(\cot\theta)' = - \csc^2\theta$.)

\medskip
    $$\int \frac{1}{x^2\sqrt{1-x^2}}\,dx = \hspace{5.0in}$$
\vfill

\clearpage\newpage
\probpts{Extra Credit}{3}  Compute and simplify the integral

\medskip
    $$\int \sec^3 \theta\,d\theta = \hspace{6.0in}$$
\vfill

\noindent \hrule

\vspace{0.2in}
\noindent You may find the following \textbf{trigonometric formulas} useful.  However, there are other trig.~formulas, not listed here, which you should have in memory, or which you know how to derive from these.

\begin{align*}
\sin(\alpha \pm \beta) &= \sin \alpha \cos \beta \pm \cos \alpha \sin \beta \\
\cos(\alpha \pm \beta) &= \cos \alpha \cos \beta \mp \sin \alpha \sin \beta \\
\sin(ax) \sin(bx) &= \frac{1}{2} \cos((a-b)x) - \frac{1}{2} \cos((a+b)x) \\
\sin(ax) \cos(bx) &= \frac{1}{2} \sin((a-b)x) + \frac{1}{2} \sin((a+b)x) \\
\cos(ax) \cos(bx) &= \frac{1}{2} \cos((a-b)x) + \frac{1}{2} \cos((a+b)x)
\end{align*}

\vspace{0.3in}

%\noindent \hrule
%\begin{center}
%\small
%\bigskip
%\textsc{blank space}
%\end{center}
%\vfill

\end{document}
