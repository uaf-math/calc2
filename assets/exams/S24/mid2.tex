\documentclass[12pt]{article}
\usepackage[top=1in, bottom=1in, left=.75in, right=.75in]{geometry}
\usepackage{amsmath}
\usepackage{fancyhdr}
\usepackage{graphicx}
\usepackage{txfonts}
\usepackage{multicol,coordsys}
\usepackage[scaled=0.86]{helvet}
\renewcommand{\emph}[1]{\textsf{\textbf{#1}}}
\usepackage{anyfontsize}
\usepackage[shortlabels]{enumitem}
% \usepackage{times}
% \usepackage[lf]{MinionPro}
\usepackage{tikz,pgfplots,mathrsfs}
%\def\degC{{}^\circ{\rm C}}
\def\ra{\rightarrow}
\usetikzlibrary{calc, backgrounds}
\pgfplotsset{compat = newest}
\newcommand{\blank}[1]{\rule{#1}{0.75pt}}

\pgfplotsset{my style/.append style={axis x line=middle, axis y line=
middle, xlabel={$x$}, ylabel={$y$},axis equal}}

%yticklabels={,,} , xticklabels={,,}

% \setmainfont{Times}
% \def\sansfont{Lucida Grande Bold}
\parindent 0pt
\parskip 4pt
\pagestyle{fancy}
\fancyfoot[C]{\emph{\thepage}}
\fancyhead[L]{\ifnum \value{page} > 1\relax\emph{Math 252: Midterm Exam 2}\fi}
\fancyhead[R]{\ifnum \value{page} > 1\relax\emph{Spring 2024}\fi}
\headheight 15pt
\renewcommand{\headrulewidth}{0pt}
\renewcommand{\footrulewidth}{0pt}
\let\ds\displaystyle
\def\continued{{\emph {Continued....}}}
\def\continuing{{\emph {Problem \arabic{probcount} continued....}}\par\vskip 4pt}

\newcommand{\prob}[1]{\bigskip\noindent\textbf{#1.} }
\newcommand{\pts}[1]{{\small (\textsl{#1 pts})}}

\newcommand{\probpts}[2]{\prob{#1} \pts{#2} \quad}
\newcommand{\ppartpts}[2]{\textbf{(#1)} \pts{#2} \quad}
\newcommand{\epartpts}[2]{\medskip\noindent \textbf{(#1)} \pts{#2} \quad}


\begin{document}
{\emph{\fontsize{26}{28}\selectfont Math F252\hfill
{\fontsize{32}{36}\selectfont Midterm II}
\hfill Spring 2024}}
\vskip 2cm
\strut\vtop{\halign{\emph#\hskip 0.5em\hfil&#\hbox to 4in{\hrulefill}\cr
\emph{\fontsize{18}{22}\selectfont Name:}&\cr
\noalign{\vskip 10pt}}}

\vspace{0.5in}
{\fontsize{18}{22}\selectfont\emph{Rules:}}

You have {90 minutes} to complete this midterm.  

Partial credit will be awarded, but you must show your work.

Calculators are not allowed. 

Place a box around your \fbox{FINAL ANSWER} to each question, or use the box provided.

Turn off anything that might go beep during the exam.

Good luck!
\vfill
\def\emptybox{\hbox to 2em{\vrule height 16pt depth 8pt width 0pt\hfil}}
\def\tline{\noalign{\hrule}}
\centerline{\vbox{\offinterlineskip
{
\bf\sf\fontsize{18pt}{22pt}\selectfont
\hrule
\halign{
\vrule#&\strut\quad\hfil#\hfil\quad&\vrule#&\quad\hfil#\hfil\quad
&\vrule#&\quad\hfil#\hfil\quad&\vrule#\cr
height 3pt&\omit&&\omit&&\omit&\cr
&Problem&&Possible&&Score&\cr\tline
height 3pt&\omit&&\omit&&\omit&\cr
&1&& 12&&\emptybox&\cr\tline
&2&&  4&&\emptybox&\cr\tline
&3&& 18&&\emptybox&\cr\tline
&4&& 18&&\emptybox&\cr\tline
&5&& 13&&\emptybox&\cr\tline
&6&& 12&&\emptybox&\cr\tline
&7&&  5&&\emptybox&\cr\tline
&8&&  6&&\emptybox&\cr\tline
&9&& 12&&\emptybox&\cr\tline
&\textsl{Extra Credit}&& \textsl{3}&&\emptybox&\cr\tline
&Total&& 100&&\emptybox&\cr
}\hrule}}}

\newpage
\prob{1}  Compute and simplify the improper integrals, or show they diverge.  Use correct limit notation.

\epartpts{a}{6} $\ds \int_0^1 \frac{dx}{x^{1/3}} =$
\vfill

\epartpts{b}{6} $\ds \int_1^\infty \frac{x\,dx}{1+x^2} = $
\vfill

% 5.1 #10, from homework
\probpts{2}{4}  Find a formula for the general term $a_n$ of the sequence
	$$\{0,3,8,15,24,35,48,\dots\} \hspace{5.0in}$$
\vspace{1.25in}


\newpage\clearpage
\prob{3} Do the following series converge or diverge?  Show your work, including naming any test you use.

\epartpts{a}{6}  $\ds \sum_{n=1}^\infty \frac{\sqrt{n+1}}{n^2}$
\vfill

\epartpts{b}{6} $\ds \sum_{n=1}^\infty \ln(n)$
\vspace{1.8in}

\epartpts{c}{6} $\ds \sum_{n=1}^\infty \frac{(-1)^n}{\sqrt{n+1}}$
\vfill


\newpage\clearpage
\prob{4} Do the following series converge or diverge?  Show your work, including naming any test you use.

\epartpts{a}{6}  $\ds \sum_{n=0}^\infty \frac{2^n}{(n+2)!}$
\vfill

\epartpts{b}{6}  $\ds \sum_{n=0}^\infty \left(\frac{n+1}{2n+3}\right)^n$
\vfill

\epartpts{c}{6}  $\ds \sum_{n=1}^\infty \frac{n}{e^{(n^2)}}$
\vfill


\newpage\clearpage
\newcommand{\threeopts}{{\small $\begin{matrix} \text{\textsc{converges}} \\ \text{\textsc{absolutely}} \end{matrix}$ \qquad\qquad $\begin{matrix} \text{\textsc{converges}} \\ \text{\textsc{conditionally}} \end{matrix}$ \qquad\qquad \textsc{diverges}} \bigskip}

\prob{5} Consider the infinite series \, $\ds 1 - \frac{1}{3} + \frac{1}{5} - \frac{1}{7} + \frac{1}{9} - \frac{1}{11} +\dots$

\epartpts{a}{4}  Write the series using sigma ($\sum$) notation.
\vfill

\epartpts{b}{4}  Compute and simplify $S_3$, the partial sum of the first three terms.
\vfill

\epartpts{c}{5}  Does the series converge absolutely, conditionally, or neither (diverge)?  Show your work, identify any test(s) used, and circle one answer. 
\vspace{3.0in}

\threeopts


\newpage\clearpage
\prob{6} Use the well known geometric series $\ds \frac{1}{1-r}=\sum_{n=0}^\infty r^n$ to find power series representations for the following functions.  Show your work.  (\textsl{Hint on part} \emph{(b)}: \textsl{Use the answer from part} \emph{(a)}.)

\epartpts{a}{6}  $\ds \frac{1}{1+x^2}$
\vfill

\newcommand{\spread}{$\ds\frac{{\large\strut}}{{\large\strut}}$}
\newcommand{\bspread}{$\ds\frac{{\Large\strut}}{{\Large\strut}}$}

\noindent \hfill \fbox{\bspread $\ds \frac{1}{1+x^2}=$ \hspace{4.0in}}
\bigskip

\epartpts{b}{6}  $\ds \arctan x$
\vfill

\hfill\fbox{\bspread $\ds \arctan x=$ \hspace{4.0in}}
\bigskip


\newpage\clearpage
\probpts{7}{5} Compute and simplify the value of the infinite series \, $\ds \sum_{n=1}^\infty \left(\frac{1}{5}\right)^{n+1}$.
\vspace{2.5in}

\probpts{8}{6} If \,$\ds f(x)=\sum_{n=0}^\infty \frac{x^n}{n!}$,\, find a \textsl{simplified} power series representation for $f'(-x^2)$.
\vfill
\hfill \fbox{\bspread $\ds f'(-x^2)=$ \hspace{4.0in}}


\newpage\clearpage
\prob{9} Find the \emph{radius} and \emph{interval} of convergence of the following power series.

\epartpts{a}{6} $\ds \sum_{n=1}^\infty \frac{3^n x^n}{n!}$
\vspace{2.5in}

\hfill \fbox{\spread $\ds R=$ \hspace{1.5in}} \qquad \fbox{\spread interval: \hspace{2.5in}}

\epartpts{b}{6} $\ds \sum_{n=1}^\infty \frac{(x+1)^n}{n\, 2^n}$
\vfill

\hfill \fbox{\spread $\ds R=$ \hspace{1.5in}} \qquad \fbox{\spread interval: \hspace{2.5in}}


\newpage\clearpage
\probpts{Extra Credit}{3} The series $\displaystyle \sum_{n=1}^\infty \frac{(-1)^n}{2n+1}$ converges to $\pi/4$.  Suppose you wanted to use this series to obtain an estimate of $\pi/4$ that is within $0.0001$ of the actual value.  Determine the fewest number of terms you would need to sum in order to obtain this level of accuracy.  Explain your reasoning.
\vfill

\noindent \hrule
\medskip
\centerline{\footnotesize \textsc{blank space}}
\vfill

\newpage\clearpage
\medskip
\centerline{\footnotesize \textsc{blank space}}
\vfill
\end{document}