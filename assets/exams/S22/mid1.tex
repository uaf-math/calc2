\documentclass[11pt]{amsart}
%\pagestyle{empty} 
\setlength{\topmargin}{-0.5in} % usually -0.25in
\addtolength{\textheight}{1.2in} % usually 1.25in
\addtolength{\oddsidemargin}{-0.95in}
\addtolength{\evensidemargin}{-0.95in}
\addtolength{\textwidth}{1.9in} %\setlength{\parindent}{0pt}

\newcommand{\normalspacing}{\renewcommand{\baselinestretch}{1.1}\tiny\normalsize}
\normalspacing

% macros
\usepackage{amssymb,xspace,alltt,verbatim}
\usepackage[final]{graphicx}
\usepackage[pdftex,colorlinks=true]{hyperref}
\usepackage{fancyvrb}
\usepackage{tikz}

\newtheorem*{lem*}{Lemma}

\newcommand{\bb}{\mathbf{b}}
\newcommand{\bc}{\mathbf{c}}
\newcommand{\bs}{\mathbf{s}}
\newcommand{\bu}{\mathbf{u}}
\newcommand{\bv}{\mathbf{v}}
\newcommand{\bw}{\mathbf{w}}
\newcommand{\bx}{\mathbf{x}}
\newcommand{\by}{\mathbf{y}}

\newcommand{\bbf}{\mathbf{f}}

\newcommand{\CC}{{\mathbb{C}}}
\newcommand{\RR}{{\mathbb{R}}}
\newcommand{\eps}{\epsilon}
\newcommand{\ZZ}{{\mathbb{Z}}}
\newcommand{\ZZn}{{\mathbb{Z}}_n}
\newcommand{\NN}{{\mathbb{N}}}
\newcommand{\ip}[2]{\mathrm{\left<#1,#2\right>}}

\renewcommand{\Re}{\operatorname{Re}}
\renewcommand{\Im}{\operatorname{Im}}

\newcommand{\Log}{\operatorname{Log}}

\newcommand{\grad}{\nabla}

\newcommand{\ds}{\displaystyle}

\newcommand{\Matlab}{\textsc{Matlab}\xspace}
\newcommand{\Octave}{\textsc{Octave}\xspace}
\newcommand{\pylab}{\textsc{pylab}\xspace}

\newcommand{\prob}[1]{\bigskip\noindent\textbf{#1.} }
\newcommand{\pts}[1]{(\emph{#1 pts})}

\newcommand{\probpts}[2]{\prob{#1} \pts{#2} \quad}
\newcommand{\ppartpts}[2]{\textbf{(#1)} \pts{#2} \quad}
\newcommand{\epartpts}[2]{\medskip\noindent \textbf{(#1)} \pts{#2} \quad}


\begin{document}
\hfill \Large Name:\underline{\phantom{Ed Bueler really really long long long name}}
\medskip

\scriptsize \noindent Math 252 Calculus 2 (Bueler) \hfill Thursday, 17 February 2022
\medskip

\LARGE\centerline{\textbf{Midterm Exam 1}}

\smallskip
\begin{quote}
\large
\textbf{No book, notes, electronics, calculator, or internet access.  100 points possible. 70 minutes maximum.}
\end{quote}

\normalsize
\medskip

\thispagestyle{empty}

\probpts{1}{8}  Compute the area between the curves $y=\cos(x)$ and $y=\cos^2(x)$ on the interval $0\le x \le \pi/2$. \quad (\emph{Hint.  Be careful about which curve is above the other.})
\vfill

\probpts{2}{6}  Completely set up, but do not evaluate, a definite integral for the length of the curve $\ds y=\frac{1}{x}$ on the interval $x=1$ to $x=10$.
\vspace{2.5in}

\clearpage\newpage
\prob{3}  Evaluate and simplify the following indefinite and definite integrals.

\epartpts{a}{6}  $\ds \int \tan x\,dx = $
\vfill

\epartpts{b}{6}  $\ds \int_0^1 3^x\,dx = $
\vfill

\epartpts{c}{6}  $\ds \int_0^{\pi/4} \tan^3 x \sec^2 x\,dx = $
\vfill

\clearpage\newpage
\epartpts{d}{8}  $\ds \int \cos^2 (7t) \sin^3 (7t) \,dt = $
\vfill

\epartpts{e}{8}  $\ds \int \cos(7t) \cos(3t)\,dt = $
\vfill

\clearpage\newpage
\epartpts{f}{8}  $\ds \int \frac{x}{x^2-4x - 5}\,dx = $
\vfill

\epartpts{g}{8}  {\large $\ds \int z^2\, e^z\,dz = $}
\vfill

\clearpage\newpage
\prob{4}  \ppartpts{a}{4} Sketch the region bounded by the curves $y=\ln x$, $x=1$, and $y=1$.
\vspace{3.0in}

\epartpts{b}{8}  Use shells to find the volume of the solid of revolution found by rotating the region in part \textbf{(a)} around the $x$-axis.
\vfill

\clearpage\newpage
\epartpts{c}{8}  Fully set up, but do not evaluate, the three integrals needed to compute the center of mass $(\bar x,\bar y)$ of the region in part \textbf{(a)} (previous page).  Then fill in the blanks at the bottom, to show how to compute the values $\bar x$ and $\bar y$.
\vfill

\newcommand{\fracboxes}{\frac{\quad \fbox{{\Huge $\strut$} \hspace{0.6in}} \quad}{\fbox{{\Huge $\strut$} \hspace{0.6in}}}}
{\large
\begin{align*}
m &= \hspace{5.0in} \\
{\Huge \strut} \\
{\Huge \strut} \\
M_y &= \\
{\Huge \strut} \\
{\Huge \strut} \\
M_x &= \\
{\Huge \strut} \\
\end{align*}

\vspace{0.5in}

  $$\bar x = \fracboxes, \hspace{1.0in} \bar y = \fracboxes$$
}

\clearpage\newpage
\prob{5}  Which trigonometric substitution would you use for the following two integrals?  Write the substitution in the box.  (There is no need to compute the integrals here.)

\epartpts{a}{4} $\ds \int \sqrt{x^2 - 16\,}\,dx$

\vspace{0.2in}

\hfill\fbox{{\Huge $\strut$} \hspace{3.0in}}
\bigskip

\epartpts{b}{4} $\ds \int \frac{t^2}{\sqrt{1 - 4t^2\strut\,}}\,dt$

\vspace{0.2in}

\hfill\fbox{{\Huge $\strut$} \hspace{3.0in}}

\bigskip\bigskip\bigskip

\probpts{6}{8}  Evaluate and simplify the integral in \textbf{5(b)} above.
\vfill

\clearpage\newpage
\probpts{Extra Credit}{3}  A donut (torus) surface is created by rotating a circle with radius one and center $(x,y)=(2,0)$ around the $y$-axis.  Fully set up, but do not evaluate, an integral for the surface area of this donut.
\vfill

\noindent \hrule

\vspace{0.2in}
\noindent You may find the following \textbf{trigonometric formulas} useful.  However, there are other trig.~formulas, not listed here, which you should have in memory, or which you know how to derive from these.

\begin{align*}
\sin(\alpha \pm \beta) &= \sin \alpha \cos \beta \pm \cos \alpha \sin \beta \\
\cos(\alpha \pm \beta) &= \cos \alpha \cos \beta \mp \sin \alpha \sin \beta \\
\sin(ax) \sin(bx) &= \frac{1}{2} \cos((a-b)x) - \frac{1}{2} \cos((a+b)x) \\
\sin(ax) \cos(bx) &= \frac{1}{2} \sin((a-b)x) + \frac{1}{2} \sin((a+b)x) \\
\cos(ax) \cos(bx) &= \frac{1}{2} \cos((a-b)x) + \frac{1}{2} \cos((a+b)x)
\end{align*}

\vspace{0.3in}

%\noindent \hrule
%\begin{center}
%\small
%\bigskip
%\textsc{blank space}
%\end{center}
%\vfill

\end{document}
